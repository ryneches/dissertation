\section{Abstract}

Diversification is the process by which a single lineage evolves and branches into a group of distinct new lineages. It is rarely the case that diversification occurs in isolation; evolving lineages interact with one another with varying intensity over time. This process is called codiversification or cospeciation when the terminal lineages are treated as separate species. When the the history of a diversifying lineage is reconstructed, it is usually represented as a tree; when the histories of interacting lineages are inferred, the interactions connect branches of the trees to one another. Such phylogenies resemble ``tangled trees.'' Most existing methods for studying interacting phylogenies fall into two categories; either they test for codiversification by fitting data to an idealized model, or in cases where codiversification is known to have occurred, they endeavour to predict the most likely history. Here, we present a comparative approach to identifying and classifying codiversification events. Using the graph spectra of networks of interacting organisms, we construct a similarity metric on the topology of their interacting phylogenies that permits the use of established clustering and machine learning techniques to classify interactions that exhibit similar phylogenetic topology and patterns of ecological interaction. This is valuable for studying host-microbe interactions in microbiomes. For many groups of host-associated microorganisms, the pattern of interaction and phylogenies of the interacting organisms are known but the nature of the ecological relationship is not. Using this method, we are able to propose ecological relationships for host-associated bacterial clades based on their structural similarity to interactions with known ecological relationships.
