\section{Materials \& Methods}

### Build OTU table, sample metadata and host tree

After much frustration, I've decided to abandon [`QIIME`](http://qiime.org/)
in favor of a simpler, more direct approach. I think this approach will be 
more useful to researchers who are pursuing novel statistical and phylogenetic
treatments of community amplicon data. 

I use [`vsearch`](https://github.com/torognes/vsearch) to perform chimera
checking and to cluster the data into OTUs. This step goes a lot faster if
a reference database of full-length 16S genes is used (I suggest the 
[`SILVA SSU Ref`](http://www.arb-silva.de/download/arb-files/) database, 
but it probably doesn't matter very much which one is used). The rest of 
the analysis is reference-free, so there is perhaps an interest in running
the chimera checking step *de novo*.

Then I use [`ssu-align`](http://selab.janelia.org/software/ssu-align/) to build an
alignment of all cluster centroids. Those who are not targeting an rRNA gene
should use an alignment method appropriate to their gene(s) of interest, 
such as [`LAST`](http://last.cbrc.jp/) for protein coding genes. Then, I build
a maximum likelihood tree with [`RAxML`](https://github.com/stamatak/standard-RAxML).

This analysis assumes you already have a tree for the host organisms. For my
study, I'm using a subset of  [rate-smoothed mega-phylogeny of actinopterygiians by
Rabosky *et al.*](http://datadryad.org/resource/doi:10.5061/dryad.j4802/1)
with a few substitutions for sister taxa that were not available.

A table of OTU counts is then constructed as a `pandas` DataFrame. Next, we load the
sample metadata as a DataFrame, which we use to drop the control samples and the other
experiments, and to merge the OTU counts for samples that have the same host species.

Last, we use rpy2 to import the host phylogeny. This is silly, but all of the tests 
for phylogenetic signal are in R, so we need to have the host tree as an R object.

Questions :
* Co-diversivtion or reciprocal adaptation?

Required software :
* [`vsearch`](https://github.com/torognes/vsearch)
* [`ssu-align`](http://selab.janelia.org/software/ssu-align/)
* [`RAxML`](https://github.com/stamatak/standard-RAxML)

Required databases :
* [`SILVA SSU Ref`](http://www.arb-silva.de/download/arb-files/)

Required python packages :
* `pandas`
* `seaborn`
* `dendropy`
* `scikit-bio`
* `ete2`