\section{Materials \& Methods}

\subsection{Specimen collection and housing}

\subsection{Sample collection}

\subsection{Sample preparation and processing}

\subsection{Sequencing}

\subsection{Building the observation table}

Adapter removal, quality trimming and overlap alignment is performed using Trimmomatic.

Chimera checking with [`vsearch`](https://github.com/torognes/vsearch)

Unique reads are identified using hat-trie

A table of observation counts is constructed as a {\tt Pandas} {\tt DataFrame}object, and a count threshold is applied. Tables of raw counts and normalized counts are written as comma separated value files, and the corresponding sequences are written as a FASTA file.

\subsection{Building phylogeny of OTUs}

Alignment of the observed sequences is performed using {\tt Clustal Omega}, and an approximate maximum likelihood phylogeny is constructed using {\tt FastTree}.

\subsection{Building phylogeny of hosts}

\subsection{Processing co-phylogenies with {\tt SuchTree}}

The host tree and guest tree are loaded as {\tt SuchTree} objects, and linked together through the observation table as a {\tt SuchLinkedTrees} object. The {\tt SuchTree} class allows for extremely rapid traversals of large trees, enabling distance correlations to be efficiently computed. The {\tt SuchLinkedTrees} class leverages this to compute graph adjacency and graph Laplacian matrixes of subtrees of host and guest phylogenies. Spectral decomposition and kernel densities of graph Laplacians are computed using {\tt numpy}, and the Jensen-Shannon divergence is calculated between each pair of spectral densities using the {\tt entropy} function in {\tt scipy.stats}. UPGMA clustering is performed using {\tt scipy.cluster.hierarchy.linkage}.

\subsection{Correlation-based analysis}

For each clade of guest organisms, the {\tt SuchLinkedTrees.linked_distances} function is used to calculate the pairwise distances through the host and guest trees for every pair of non-null observations in the link matrix, as described by Hommola {\em et al.} \cite{hommola2009permutation} The Pierson's correlation for these distances is computed using the {\tt pearsonr} function from {\tt scipy.stats}.

\subsection{Comparative analysis}

