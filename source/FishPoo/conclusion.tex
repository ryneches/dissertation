\section{Conclusion}

It is difficult to overstate the importance of the role that intuition plays in guiding research. While it does not appear within a rigorous experiment, it nevertheless frames it. Intuition attracts researchers to take interest in some things and not others. Intuition prompts researchers to pose the questions that are ultimately refined into testable hypotheses. Intuition guides the priorities researchers apply when they decide which hypothesis is worthy of rigorous testing. Good intuition is as much a part of the scientific process as the careful logical reasoning that ultimately sustains or rejects the hypothesis. The difficulty is that intuition is heuristic, and heuristics draw on experience. The great majority of the complexity to be found in microbial ecology exists outside of what can be experienced directly, and so researchers often struggle to frame their work.

The approach we present here does not directly test the interactions it predicts. Rather, it is a powerful way to draw on experiences with other ecosystems to refine our intuition about microbial ecosystems. 