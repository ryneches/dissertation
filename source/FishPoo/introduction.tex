\section{Introduction}

The dispersal of organisms through their habitat is the unifying theme for the natural history of both the organism and the habitat. Dispersal determines whether or not organisms will meet and interact, and if so, how often; it shapes the nature of interactions that do occur. Dispersal is the first-order effect behind the establishment of species ranges, and the interpenetrating dispersal of many organisms is the process that drives the assembly of ecosystems. However, dispersal has only been thoroughly studied in a handful of organisms, consisting mostly of plants, animals and diseases of plants and animals. Relative to plants and animals, little is known about the dispersal of microorganisms, and as a result, very little is understood about their natural histories. 

This dearth of knowledge is due to a lack of observational tools with sufficiently broad scope and detailed resolution. Microscopy, culture-based and molecular genetic methods have revealed a great deal about the physiology of microorganisms, but they have revealed only a patchy, distorted view of microbial natural history. Consider, for example, what inferences a well-trained naturalist could draw if she noticed a bird within her habitat of expertise. She would likely know whether or not it is unusual to find in that habitat. She would know whether it is migratory or resident. In many cases, she would be able to guess its approximate origin with some confidence. If she instead discovered a bacterium, once our naturalist has gone to a good deal of trouble and expense to identify the organism, she would be somewhat lucky if the closest relative known to science could be said to be within the same genus. With the available tools, there is just as much taxonomic uncertainty with respect to canonical references as among one's own previous observations. Not only would it be difficult for her to know which species she was looking at, but it would be uncertain if previous, identical observations were, in fact, the same species. Given all of this uncertainty, it is in most cases difficult to say if an observation is unusual or surprising. As for where it came from, she would be forced to speculate. Clearly, our naturalist needs better tools with which to observe and probe the natural history of the microbial denizens of her system.

In nearly every ecosystem to which it has been applied, culture-independent sequencing has revealed massive diversity of microorganisms. There has been progress in characterizing the biodiversity of many microbial ecosystems by coupling these tools with databases and statistical models. For some ecosystems, such as the human gut, geothermal springs, hypersaline lakes, the cow rumen and some fermented foods, it might be said that the biodiversity is reasonably well characterized. In those systems, ecologists have begun to shift their focus from surveying biodiversity to characterizing interactions, and the question ``Who is there?'' is beginning to give way to ``What are they doing?'' This represents important progress, but bypasses the question, ``Where did they come from?'' This is unfortunate, because continuity through different contexts is often the strongest evidence signifying an organism's ecosystem function.

--

Although motion through space is conceptually simple, the movement of populations of organisms is a complex a nuanced problem. To make it more tractable, we will decompose it into smaller problems. 

The first problem that must choose a coordinate system. The most obvious coordinate system is space; geographic distance often serves as a barrier separating populations. Among plants and animals, it is natural to conceptualize dispersal in geographic terms. A {\em species range}, as it is commonly understood by ecologists, is something that is drawn upon a map. 

There has been some excellent work to develop biogeography of microbes, particularly among extremeophiles. Rachel Whitaker and Thane Papke have challenged the Bass Becking model by looking at the biogeography of thermophilic microbes (such as {\em Sulfolobus islandicus} and {\em Oscillatoria amphigranulata}), first by 16S rRNA phylogenetics and later using high resolution, multi-locus methods. Both Whitaker's work and Papke's work, as well as many studies of disease evolution, demonstrate that when you look within a microbial species, the populations do not appear quite so cosmopolitan. While {\em Sulfolobus islandicus} is found in hot springs all over the world, the evolutionary distance between each pair of its isolates is strongly correlated with the geographic distance between their sources. So, these microbes are indeed getting around the planet, but if we look more closely, we see that they are not getting around so quickly. 

Nevertheless, extremeophiles are amenable to this approach because their relationship with geographic features is unusual among microbial organisms. Most microbes are found in distributions that have a less obvious relationship to geography. For example, planktonic communities in the ocean are rather undifferentiated along latitude and longitude, but differ sharply along depth and season. In soil, community composition is more strongly influenced by physical properties like pH and humidity than by geography.

It is from these observations, and others like them, that Lorens Baas Becking proposed what has become the most successful conceptual model for microbial community formation : {\em everything is everywhere, but the environment selects}\footnote{{\em Alles is overal: maar het milieu selecteert.}}. This is a profound idea; it asserts that microbial dispersal is effectively infinite, and that differences in the composition of microbial communities is due to selection alone. The phenomenon of sites that seem identical but have different communities is explained as a failure to understand and measure their selective properties well enough.

The Baas Becking model exhorts the researcher to study the context in which microorganisms are observed. Classical microbiology is entirely built on this idea; if we apply macroscopic tools to microscopic organisms, we are perforce working through environmental selection. This has proved to be a powerful tool for microbiology, and much of what we know about microbial physiology, microbial diversity and community structure has been learned by careful control of selection. 

Nevertheless, it is important to recognize that the Baas Becking model cannot be literally true; it is only true to within the limits of the observational methodology. Microbes do not disperse among the continents by quantum teleportation; they must face barriers and obstacles, some perhaps insurmountable, as well as conduits and highways. Even with their rapid growth and vast numbers, this landscape of barriers and conduits must influence their propagation. Of course, Baas Becking was aware of this. In the same 1934 book in which he proposed his eponymous hypothesis, Baas Becking observed that there are some habitats that were ideally suited for one microbe or another, and yet these microbes were not present. He offered the following explanation: ``There thus are rare and less rare microbes. Perhaps there are very rare microbes, i.e., microbes whose possibility of dispersion is limited for whatever reason.''

A monolithic population -- one in which all players are in the same compartment -- evolves differently than a fragmented population, even if mutation, recombination and selection pressures are identical. Bass Becking's hypothesis is a statement about the nature of this structure, specifically, that the structure is monolithic. If true, it means that the only difference between an Erlenmeyer flask and the entire planet is the number of unique niches. It is an assertion that scale is irrelevant.



--

Classifying organisms by how long they have been present in a particular habitat can yield insights regarding their origins, especially when these relationships are repeated in many similar but distinct habitats. Phylogeographic models regard this collection of habitats as the nodes of a meta-community network. The edges of this network represent dispersal rates, a convolution of both the vectors of dispersal and the barriers that impinge on those vectors. Organisms move from habitat to habitat in discrete events with probabilities related to the dispersal rates. If organisms are gathered from each habitat, and trees of related organisms are constructed, these trees may be used to infer the rates of dispersal. These rates reveal, at least in broad strokes, the structure of the relationships among the habitats.



In this study, I attempt to use current technology to answer this and some related questions. As technology allows researchers to gather more phylogenetically informative data, the power of the approach described here will improve. 

The assemblage of habitats I have chosen for this study are the Tanganyikan cichlid fish. With the exception of {\em Tylochromis polylepis} and {\em Oreochromis tanganicae}, the Tanganyikan cichlids are a species flock of more than two hundred described species sharing a common ancestor dating to the formation of the lake between nine and twelve million years ago. While some members of this species flock are older than hominids, other lineages have undergone much more recent speciation, including some in which the process of adaptive radiation appears to be ongoing. Within this species flock, a large variety of reproductive, defense and trophic strategies have evolved. Other than the anoxic abyssal region, cichlids have colonized every habitat in the lake and occupy every trophic level above primary producers. Taken together, this species flock represents roughly 1\% of all described fish species and 0.5\% of all vertebrate species.

The cichlid radiations of the African Great Lakes (Tanganyika, Victoria, Malawi, Alber, Kyoga, Edward and Kivu) are 