\section{Introduction}

Ecology is the study of the relationships among organisms and between organisms and their environment. When making observations in a macroscopic system, a trained observer will find a vast amount of information in plain sight. Even if the scope of the observational protocol is defined narrowly, other potentially important information is nevertheless available. Behavior will be displayed. Aspects of life cycles may appear. Abiotic properties of the habitat, such as elevation and physical structure, will be known. Ecological connections with other organisms outside the scope of the protocol may present themselves. These pieces of data provide the observer with a great deal of context about the system under examination, and may help the identify weaknesses in the observational protocol, in the overall study design or in the hypothesis itself. Often, such contextual information only finds its way into rigorous experimental design by helping to frame and focus the hypothesis. 

% I hope this illustrates the idea more clearly now. I'm interested the kind of information that is often NOT explicitly mentioned anywhere in the results that people publish. For example, suppose you have a hypothesis like, "The role of tails in sexual displays of birds and mammals is an example of convergent evolution." To frame a hypothesis like that, at some point you probably had to have a thought like, "Huh. I just noticed that birds and mammals both have tails."

% If you're missing this sort of information, it's easy to come up with totally uninformative frames that look perfectly reasonable.

% I'm trying to argue that in microbiology, this information doesn't fall into your lap in the way it often does in other areas of biology. Visual observations are interesting and important, but because you can't connect visual observations to taxonomy very well, they don't play the same role in microbiology that they do in other areas. Instead of being able to say ``birds and mammals both have tails,'' you would have to say, ``A lot things have tails.'' You can't get to the idea about convergent evolution from there.

% I know this point seems pretty abstract, but I want to argue that the tool I've built helps bring some of this ``accidental'' knowledge to microbial ecology. 

While it is well established that microorganisms exist in complex relationships with their environment and with one another, for the overwhelming majority of microorganisms known to science, our only direct knowledge are the sequences of marker genes, from which one may posit where the organisms fall in the tree of life. As with macroscopic communities, one can also make non-targeted observations of microbial communities using microscopy and environmental sequencing. However, one often cannot cross-reference or link these observations to one another in the intuitive way that is usually possible in macroscopic communities. In a macroscopic community, the link between molecular data and observations of morphology, behavior and life history is often an inevitable consequence of the need to physically collect the sample from individual specimens. In microbial communities, establishing such links is painstaking, deliberate work, and is often impractical.

Even so, there are other complications that tend to be much more pronounced in microscopic communities. Metabolically intimate relationships are often separated over vast distances and long intervals, and mediated by substances that cannot be observed optically and require sensitive, targeted chemical analysis to measure. Convergent evolution often drives distantly related organisms to present indistinguishable morphology, and closely related organism frequently present highly divergent morphology. The great diversity of microorganisms routinely outstrips our ability to collect, organize and interpret meaningfully specific metadata.

The result is that knowledge of microbial communities tends to be much more fragmented than macroscopic communities, even when the total amount known is roughly the same. Key insights into microbial ecology are often obscured by methodological ambiguities that confound cross-referencing different categories of observational data. 

Coevolution, the reciprocal evolutionary responses arising from interactions among (at least) two distinct populations, can serve as a point of leverage for building a more integrated understanding of ecological communities. Different types of ecological interactions impose distinct pressures on the evolution of the organisms they involve, which may leave patterns in the structures of the phylogenies of the interacting organisms that can be identified if the right conditions hold. \cite{gibson2015phylogenetic, bergstrom2003red, gokhale2012mutualism} Of course, not all interactions can be inferred from the effect they have on the evolution of the organisms involved. As Janzen argued, coevolution is not synonymous with `interaction' or `symbiosis' or `mutualism,' and applies only to the subset of interactions that drive reciprocal evolutionary responses. \cite{janzen1980coevolution} In order for the patterns created by evolutionary responses to be observable, the interaction must persist for long enough to produce evolutionary responses, and those responses must be large enough to be picked up by the phylogenetic markers and methods applied.\footnote{For a more thorough but less formal exploration of the uncertainties that come into play when linking phylogenies to other processes and to other phylogenies, please see the Main Introduction.}

Although not all interactions that have a significant impact on the fitness of an organism lead to coevolution, coevolution is always the outcome of an ecological process that has a significant impact on the fitness of an organism. Organisms in predator-prey, host-parasite and other antagonistic interactions evolve with Red Queen dynamics \cite{van1973new, gibson2015phylogenetic}, where reciprocal adaptive responses drive cospeciation. Mutualistic interactions tend to obey different dynamics. \cite{Hillesland2017} The Red King model, \cite{bergstrom2003red, gokhale2012mutualism} for example, predicts that in mutualistic interactions, advantages tend to accrue to slower-evolving organisms. This has the effect of reinforcing the development of complex, highly nested interactions. \cite{bastolla2009architecture, rezende2007non} This is by no means an exhaustive list of types of interactions or the evolutionary dynamics they drive. 

If ecology can drive evolutionary processes, then it ought to be possible in some cases to infer the nature of ecological interactions from the signature they imprint on the evolution of the organisms involved. Such inferences are potentially of great value for understanding microbial communities, where marker gene sequences and place of origin are usually the only pieces of data available for the bulk of observed biodiversity.

Most methods for predicting ecological function rely on correlations with a putative mechanism which is either observed directly (as in macroscopic systems) or inferred through proxies such as small molecules, \cite{larsen2015predicting, sardans2011ecological, bundy2009environmental} annotated gene functions, \cite{jiang2012functional, bik2014deciphering} gene regulatory reponses, \cite{mason2012metagenome, urich2008simultaneous, gifford2013expression} or protein sequence profiles. \cite{wang2016environmental} Among macroscopic organisms, identification of the mechanism of an interaction is usually carried out from visual observations. For example, a wasp using an ovipositor to lay eggs inside the body of a caterpillar is unlikely to be anything but a parasitic interaction. For interactions involving microbial organisms, one might infer that an interaction is parasitic through the observation of a molecular mechanism, such as a Type III secretion system. Type III secretion systems play a key role in the pathogenic behavior of several organisms, and one may reason that a homologous system has similar function in a new context, and so when the presence of a Type III secretion system is inferred from its DNA, RNA or protein sequence, or if its activity is inferred from transcription levels of its component genes, this may indicate that one is observing a parasitic interaction. Here, we describe the development and testing of a novel approach, which we call TangleSpace, that is agnostic to the mechanism, and instead relies on observing the evolutionary effect of occupying an ecological niche.

We conducted a survey of the literature to gather examples of interactions among groups of organisms, and labeled them according to the type of interaction (parasitism or mutualism). We were able to further subdivide mutualistic interactions into pollination and frugivory (more subdivisions are possible with more data). For each of these interactions, we obtained phylogenetic trees of the interacting organisms, or built them when they were not available. We also simulated interactions with no co-evolution and with perfect co-evolution to serve as null hypotheses. Using this database of interactions with known, labeled ecologies, we use machine learning to make predictions about the ecological roles of microbial clades in our system. This approach will work for any system where the microbial community is distributed among several host species, but it may be easily generalized to any system where dispersal through the system and the phylogenetic divergence exhibit significant, detectable variation on similar time scales.

The habitat we have chosen to illustrate this method is a group of 14 species of Tanganyikan cichlid fish. With the exception of {\em Tylochromis polylepis} and {\em Oreochromis tanganicae}, the Tanganyikan cichlids are a species flock of more than two hundred described species sharing a common ancestor dating to the formation of the lake between ten and twenty million years ago. While some members of this species flock are older than hominids, other lineages have undergone much more recent speciation, including some in which the process of adaptive radiation appears to be ongoing. Within this species flock, a large variety of reproductive, defense and trophic strategies have evolved. Other than the anoxic abyssal region, cichlids have colonized every habitat in the lake and occupy every trophic level above primary producers. Taken together, this species flock represents roughly 0.5\% of all described fish species and 0.25\% of all vertebrate species. To interrogate the microbiomes of specimens of these 14 representative species, we performed 16S rRNA gene sequencing of stool samples and built a phylogenetic tree of all unique, non-chimeric sequences. By linking this tree with the host phylogeny through the presence/absence matrix of observed sequences, we created a ``tanglegram'' representing all putative coevolutionary events between the hosts and their associated organisms, as represented by their rRNA gene sequences. Because the host tree represents organisms that emerged much more recently than many of the deeper branches in the microbial rRNA tree, we searched for coevolutionary events by examining interactions with sub-clades of the rRNA tree.