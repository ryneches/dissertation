\section{Introduction}


In this study, we attempt to use current technology to answer this and some related questions. As technology allows researchers to gather more phylogenetically informative data, the power of the approach described here will improve. 

The assemblage of habitats we have chosen for this study are the Tanganyikan cichlid fish. With the exception of {\em Tylochromis polylepis} and {\em Oreochromis tanganicae}, the Tanganyikan cichlids are a species flock of more than two hundred described species sharing a common ancestor dating to the formation of the lake between ten and twenty million years ago. While some members of this species flock are older than hominids, other lineages have undergone much more recent speciation, including some in which the process of adaptive radiation appears to be ongoing. Within this species flock, a large variety of reproductive, defense and trophic strategies have evolved. Other than the anoxic abyssal region, cichlids have colonized every habitat in the lake and occupy every trophic level above primary producers. Taken together, this species flock represents roughly 0.5\% of all described fish species and 0.25\% of all vertebrate species.