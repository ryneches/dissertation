\section{Introduction}

The dispersal of organisms through their habitat is the unifying theme of the natural history of both the organism and the habitat. Dispersal determines whether or not organisms will meet and interact, and if so, how often; it shapes the nature of interactions that do occur. Dispersal is the first-order effect behind the establishment of species ranges, and the interpenetrating dispersal of many organisms is the process that drives the assembly of ecosystems. However, dispersal has only been thoroughly studied in a handful of organisms, consisting mostly of plants, animals and diseases of plants and animals. Very little is known about the dispersal of microorganisms, and as a result, very little is understood about their natural histories. 

This dearth of knowledge is due to a lack of observational tools with sufficiently broad scope and detailed resolution. Microscopy, culture-based and molecular genetic methods have revealed a great deal about the physiology of microorganisms, but they have revealed only a patchy, distorted view of microbial natural history. Consider, for example, what inferences a well-trained naturalist could draw upon discovering a species of bird within her habitat of expertise, as opposed to a species of bacillus. For the bird, she could tell you instantly whether or not it is unusual to find in that habitat, whether it is migratory or resident, and in many cases would be able to hazard a guess as to its approximate origin with some confidence. For the bacillus, once our naturalist has gone to a good deal of trouble and expense to identify the organism, she would be somewhat lucky if the closest relative known to science could be said to be within the same species. If she had not seen that organism before, it would be hard for her to say whether or not it is unusual or surprising to have found it. As for where it came from, speculation is all she could offer. Clearly, our naturalist needs better tools with which to observe and probe the natural history of the microbial denizens of her system.

In nearly every ecosystem to which it has been applied, culture-independent sequencing approaches have revealed an almost comical diversity of microorganisms. By coupling these tools with databases and statistical models, there has been progress in characterizing the biodiversity of many microbial ecosystems. For some ecosystems, such as the human gut, geothermal springs, hypersaline lakes, the cow rumen and some fermented foods, it might be said that the biodiversity is reasonably well characterized. In those systems, ecologists have begun to shift their focus from surveying biodiversity to characterizing interactions, and the question ``Who is there?'' is beginning to give way to ``What are they doing?'' This represents important progress, but leaves unanswered the question, ``Where did they come from?'' In this study, I attempt to use current technology to answer this and some related questions in a rough sense. As technology allows researchers to gather more phylogenetically informative data, the power of the approach described here will improve.

Classifying organisms by how long they have been present in a particular habitat can yield insights regarding their origins, especially when these classifications are repeated in many similar but distinct habitats. Phylogeographic models regard this collection of habitats as the nodes of a meta-community network. The edges of this network represent dispersal rates, a convolution of both the vectors of dispersal and the barriers that impinge on those vectors. Organisms move from habitat to habitat in discrete events with probabilities related to the dispersal rates. If organisms are gathered from each habitat, and trees of related organisms are constructed, these trees may be used to infer the rates of dispersal. These rates reveal, at least in broad strokes, the structure of the relationships among the habitats.

The assemblage of habitats I have chosen for this study are the associated gut microflora of Tanganyikan cichlid fish. With the exception of {\em Tylochromis polylepis} and {\em Oreochromis tanganicae}, the Tanganyikan cichlids are a species flock of more than two hundred described species sharing a common ancestor dating to the formation of the lake between nine and twelve million years ago. While some members of this species flock are older than hominids, other lineages have undergone much more recent speciation, including some in which the process of adaptive radiation appears to be ongoing. Within this species flock, a large variety of reproductive, defense and trophic strategies have evolved. Other than the anoxic abyssal region, cichlids have colonized every habitat in the lake and occupy every trophic level above primary producers. Taken together, this species flock represents roughly 1\% of all described fish species and 0.5\% of all vertebrate species.

The cichlid radiations of the African Great Lakes (Tanganyika, Victoria, Malawi, Alber, Kyoga, Edward and Kivu) are 