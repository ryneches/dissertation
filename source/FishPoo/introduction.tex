\section{Introduction}

Ecology is the study of the relationships of organisms among each other and with their environment. When making field observations in a macroscopic system, a vast amount information can be gathered simply by looking. Such contextual information may only find its way into rigorous experimental design by helping to frame and focus the hypothesis, but often that is crucially important.

While is well established that microorganisms exist in complex relationships with their environment and with one another, for overwhelming majority of microorganisms know to science, the only thing known is the simple fact of their existence. One can also look at microscopic systems, but microscopy is not equivalent to field observations of macroscopic systems. Many crucial microbial interactions are chemical in nature. Metabolically intimate relationships are often separated over vast distances and long intervals, and mediated by substances that cannot be observed optically. Convergent evolution often drives distantly related organisms to present identical morphology, and phenotypic plasticity allows closely related organism to present highly divergent morphology. The great diversity of microorganisms routinely outstrips our ability to collect, organize and interpret meaningfully specific metadata. Drawing inferences about relationships without an intuitive way to observe behavior -- often without any intuition at all -- is the central challenge of microbial ecology. 

We offer a solution to this problem that leverages structural features of interacting phylogenies to make predictions about the ecological roles of microbes. In this study, we present a version of this approach that will work for systems where the microbial habitat ramifies among several host species, but it may be trivially generalized to any habitat structure where rate of dispersal through the habitat is similar to the rate of microbial phylogenetic divergence that can be detected.

The habitat we have chosen for this study are a sample of 14 species of Tanganyikan cichlid fish. With the exception of {\em Tylochromis polylepis} and {\em Oreochromis tanganicae}, the Tanganyikan cichlids are a species flock of more than two hundred described species sharing a common ancestor dating to the formation of the lake between ten and twenty million years ago. While some members of this species flock are older than hominids, other lineages have undergone much more recent speciation, including some in which the process of adaptive radiation appears to be ongoing. Within this species flock, a large variety of reproductive, defense and trophic strategies have evolved. Other than the anoxic abyssal region, cichlids have colonized every habitat in the lake and occupy every trophic level above primary producers. Taken together, this species flock represents roughly 0.5\% of all described fish species and 0.25\% of all vertebrate species.

Our method uses techniques from graph theory to construct a high dimensional space representing structural features of networks of interacting host and microbe phylogenies. Using a database of interactions with know ecological roles gathered from the literature, we use standard machine learning techniques to make predictions about the ecological roles of microbial clades in our system.