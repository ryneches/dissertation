\section{Results}

The construction of a feature space of topological properties of interactions and dissimilarities with respect to interactions of known ecologies makes it possible to recast microbial ecology as a machine learning problem. This sketches out a powerful and flexible framework for extracting inferences on the nature of many kinds of ecological interactions without direct observation of their mechanism. The cost is that one must assemble a collection of relevant training data, and it is limited to cases where the interaction has persisted long enough to leave an significant imprint in the evolutionary history of the interacting groups. It should work particularly well in cases where two or more adaptive radiations have interacted.

The training problem can be visualized by selecting a subspace spanned by two axes of the feature space and projecting the labeled training data and the unlabeled experimental data into it. Alternatively, one can project into a subspace spanned by principle components. Similarly, the predictions can be visualized by projecting the experimental data into one of these subspaces with labels corresponding to the predictions (Figure \ref{fig:FP_classified}).

\subfile{FishPoo/figure13}

This study is limited by the small number of labeled interactions we were able to extract from the literature (50 interactions from the literature and 100 simulated interactions). With a training set of suitable size, the machine learning process calls for the refinement of the classifier by splitting the training data into training and testing sets. The classifier should be trained using the training set and its performance scored using the testing set. The tuning parameters of the classifier would be adjusted, re-trained and re-scored using multidimensional gradient descent to minimize the classification error. Here, we show only a single iteration of this process using a trained but unoptimized neural network as our classifier. Our trained neural network predicts the correct labels for the interactions with which it was trained about 97-98\% of the time (there is some stochastic variation), but this does not represent a rigorous examination of its accuracy on unlabeled data. Fortunately, our training set represents a negligible fraction of the interactions found in the ecology literature. The effort of extracting, reformatting and standardizing it is the only thing that limited us to 50 examples. An automated approach is under development, but is beyond the scope of this study.