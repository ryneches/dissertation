\section*{Introduction}

When I began reading the literature on genomic and environmental DNA sequencing in late 2006, the field was incandescent with excitement. For researchers exploring the dark landscape of microbial ecology with flashlights (classical methods) and laser range-finding (genetic methods), the appearance of culture-independent methods was like the rise of a crescent moon. For the first time, it was possible to see the shape of the land, if not its true color and full detail. It was a profound change in perspective. The community was coming to terms with the fact that the most abundant organism on the planet -- SAR11 -- had gone unnoticed until environmental rRNA surveys detected it in 1990. It was not isolated in pure culture until 2002. Around the time I was grappling with the vocabulary of a new field, the {\em Sorcerer II} circumnavigated the globe and fired a broadside of metagenomes into the public databases. \cite{yooseph2007sorcerer, rusch2007sorcerer, williamson2008sorcerer} While these methodological approaches were beginning to take hold, the shift to high throughput sequencing platforms ignited a dizzying acceleration in the quality, depth and scope of sequencing projects. When I arrived in the laboratory, I was asked to join a DARPA program tasked with uncovering {\em The Fundamental Laws of Biology}.\footnote{To my knowledge, no such laws were uncovered, but I did contribute to a fruitful collaboration where we reanalyzed the {\em Sorcerer II} data. It appears here as chapter 1 in {\em Part II: Oceans}.} The jetwash of accelerated discovery blew everyone back into the classroom. As a new student of biology, I found myself in august company.

Things have cooled somewhat, though the pace of discovery continues to accelerate. The limits of culture-independent methods are better understood, or at least better appreciated. While it revolutionized our understanding of the extent, the dynamics and the distribution of microbial diversity, the new methods and analytical tools shed relatively little light on the roles organisms play in their ecosystems. We now have a much clearer and more complete picture of what organisms are present, but knowledge of what they do has not kept pace. In proportion to what we see of the community, we now know less about behavior that we did before culture-independent methods. The uneven pace of progress is due to the slow improvement in our ability to predict a gene's function from its sequence, and from the fact that it remains difficult to measure gene activity across a community in a culture-independent manner.

In retrospect, this is not surprising. In macroscopic systems, it is usually impossible to learn anything about ecosystem function from a composite of very small details. An inventory of myoglobin sequences will not reveal whether bears typically eat salmon or salmon typically eat bears. However, knowledge of the physiology of bears and salmon, and how their respective physiology drives their ecosystem roles {\em does} reveal interesting insights on myoglobin.

Sound reasoning must draw inferences {\em from} areas of understanding {\em to} areas of ignorance. We know very little about a very small number of molecular mechanisms, and so most sequence data makes a poor anchor from which to begin building chains of inference. Rather, its value is often as a target one may reason {\em towards}. The dream of molecular biology and molecular ecology is a sort of epistemological hysteresis; macroscopic observations form the foundation for inferences on molecular features, which then form the foundation for inferences on macroscopic features, and back again, weaving a balustrade across the gap of scale and scope. As yet, there are but few such bridges. The sequencing revolution has not been accompanied by similar revolutions in other data collection methods, and so we find ourselves awash in a vast effusion of answers to questions we do not know how to ask.

My task, as I have seen it, has been to marshal resources that can help navigate this inverted intellectual space. Rather than attempting to use DNA sequencing to understand the world, I have sought contexts that yield insights into sequencing data. If nothing in biology makes sense except in the light of evolution, then nothing in evolution make sense except in the light of the systematic comparison of its manifestations. I offer here a powerful new approach that lays open microbial ecology to comparative applications of machine learning, and I chronicle the steps I made toward that goal.

In the following pages, I will present eleven of these enterprises. With two exceptions, these projects have all been brought to conclusion and offered into the scientific record. Many of these projects look rather different upon their conclusion than they did at initiation. The common thread among them is perhaps difficult to see retrospectively, although it was clear enough before the realities of the research process metabolized our ideas into publications. Before moving into the chapters, I will do my best to trace the common threads that run through them -- or near them, anyway.

I have organized these enterprises into eleven chapters in five parts. Nine of these chapters represent published work, and I have provided descriptions of author contributions for each of them. There are many themes that link them, and many defensible ways they could muster to the parade grounds you now hold in your hands. The formation I have chosen is a typical improvisation.

In {\em Part I: Fish}, I present three chapters in which the relationships among an adaptive radiation of cichlid fishes and their microbial symbionts are explored. The first chapter represents the (as yet unpublished) culmination of largest effort and most important enterprise that appears in this dissertation, and lays out a novel method for predicting ecological interactions among microbes and their associated hosts. The second chapter explores gene flow among the hosts, and the third chapter explores the ecological dynamics of a mass extinction among the hosts.

In {\em Part II: Oceans}, I present three chapters exploring the biogeography of the planktonic microbial ecology of the open ocean. The first chapter develops a statistical approach for predicting ecological functions of protein-coding genes observed from metagenomic data and applies this approach to the 2004--2007 Global Ocean Survey. Chapters two and three present results from a microbial survey of the Indian Ocean to which I contributed hardware for measuring basic water chemistry.

In {\em Part III: ChIP-seq}, I present two chapters that develop the necessary tools for exploring the gene regulatory networks of halophilic archaea.

In {\em Part IV: Tool}, I present two chapters that develop applications for 3D printing in microbiology and molecular biology. In the first chapter, I establish that fused deposition modeling 3D printing is alone sufficient to produce sterile components. In the second chapter, I present single-use 3D printed parts, and the software to create them, that can perform the role of a liquid handling robot. 

In {\em Part V: Space}, I present a single chapter that relates a basic microbial growth kinetics experiment aboard the International Space Station.

\subsection*{An interlude}

Before I proceed into these chapters, I would like to share some thoughts that have preoccupied me throughout this work. Originally, I came to graduate school to study plasma physics, which I found captivating for the way that ordered dynamics emerge from the chaotic interactions of vast numbers of small actors. When I began exploring the idea of pursing microbial ecology, it was with this frame of mind. Just as plasma physics soars over a classically unsolvable many-body problem on the crooked wings of statistical mechanics, ecology makes a similar flight over the domains of metabolism, reproduction and trophism. The wings aren't quite the same, but they are feathered with the same kinds of differential equations and asymptotic orderings. There is something exhilarating about moving from the singular to the continuous without getting snared by the gibbering madness of the unanalyzable collective that lays between. It feels like magic, and I love it because it isn't.

Gnarled though it is, the laws of thermodynamics run through statistical mechanics in a triple braid of truth. Of the three, I usually prefer the first -- conservation of energy -- as my polestar, and that was how I planned to continue in microbial ecology. 	``Surely,'' I reasoned, ``if the laws of thermodynamics guide molecular biologists through the nightmare landscape of cellular metabolism, then why shouldn't they guide me to someplace interesting in microbial ecology?'' My efforts in this pursuit appear in {\em Part III: ChIP-seq}. Before I could get very far in that pursuit, my attention was seized by a much simpler question.

I had once found myself standing with Dr. Karl Stetter about an hour's climb below the crater of Mutnovsky volcano. A short distance from the trail he had noticed a column of what turned out to be acrid, stinging fog. We followed it to the source, and found a tiny, roiling microcosm of seemingly alien life. A thermal spring. I was curious about how the community was organized through the exchange of parcels of energy, and I recognized in the moment an opportunity to shake loose some wisdom from one of the world's eminent experts on such things. However, I was surprised to discover that this wasn't what I wanted to know about. I think this was perhaps the moment that marked the close of my career as a physicist, and the opening of other worlds.

The prospect surrounding the little spring was littered with freshly cooled boulders of tephra crenelated in prismatic flags of rime ice. In this vista, a pecuniary audit of joules and calories seemed unimportant and, oddly, somehow crass. The ecosystem was about the size of a bowl of soup, and was cradled in a small depression in the regolith. I could have scooped it up and held it in my hands if I hadn't minded scalding myself. A firm nudge with the toe of my boot would have obliterated it. Gases from the volcano bubbled up from a tiny vent in the center of the puddle, gently stirring a biofilm that resembled overcooked rice noodles. Like many other springs, the cooler rim was foamy pink encircled by a narrow annulus of slimy green. Encircling that was a raised gasket -- a tiny boneyard of dead, dessicated things. Beyond the boneyard, the residents would find freezing, starving death stretching hundreds of kilometers in every direction. 

Just like every other community, the organisms in that tiny spring balanced their thermodynamic ledgers, and their society was thus shaped by the inescapable arithmetic of nutrition, energy and waste. Nevertheless, I found that I could not convince myself that this was the most important thing I was seeing. In that moment, questions about electron transport chains and redox potentials felt disrespectful of their story. Their real story, it seemed, was how they came to be there at all.

One should avoid anthropomorphizing non-human creatures, of course. It is easy to project human traits onto things that do not have them, and from the mixture of real and imagined characteristics to draw conclusions that have no basis in fact. In metaphorical terms, though, sometimes an anthropomorphic perspective can aid the imagination if one is mindful not to draw conclusions. For the microbes in little spring, I couldn't help but wonder what their history would be like if it were told in human terms. The journey from another little spring across such a vast wasteland would make a story belonging among our great epics. What catastrophe drove them from their former home? What fiery furnaces did they endure? What virtues did they possess that allowed them pass through the lion's den?

I asked Dr. Stetter how he imagined this epic might have gone (though not in those terms). He answered with a shrug. Perhaps they came on the feather of a bird, or on the claw of a bear? Perhaps they were borne on the fickle wind, or trafficked by a network of secret subterranean plumbing? Who could know, and how?

This question took hold of me, then and there, and I have not been able to shake it. With some reluctance, I decided to leave the inner lives and great societies of microorganisms to other people. You will find no discussion of Boltzmann or Gibbs in the pages that follow. What you will find is my search for the sagas, epics and legends of the heroic voyages of our smallest cousins. I want to read their natural history.

\subsection*{On the problem of dispersal}

In nearly every ecosystem to which it has been applied, culture-independent sequencing has revealed a massive diversity of microorganisms. There has been progress in characterizing the biodiversity of many microbial ecosystems by coupling these tools with databases and statistical models. For some ecosystems, such as the human gut, geothermal springs, hypersaline lakes, the cow rumen and some fermented foods, it might be said that the biodiversity is reasonably well characterized. In those systems, ecologists have begun to shift their focus from surveying biodiversity to characterizing interactions, and the question ``Who is there?'' is beginning to give way to ``What are they doing?'' This represents important progress, but bypasses the question, ``How did they get there?'' This is unfortunate, because continuity through different contexts is often the strongest evidence signifying an organism's ecosystem function.

Dispersal determines whether or not organisms will meet and interact, and if so, how often. It shapes the nature of interactions that do occur. It is the first-order effect behind the establishment of species ranges. The interpenetrating dispersal of many organisms is the process that drives the assembly of ecosystems. However, dispersal has only been thoroughly studied in a handful of organisms, consisting mostly of plants, animals and diseases of plants and animals. Relative to plants and animals, little is known about the dispersal of microorganisms, and as a result, very little is understood about their natural histories. Microscopy, culture-based and molecular genetic methods have revealed a great deal about the physiology of microorganisms, but they have revealed only a patchy, distorted view of microbial natural history.

Consider, for example, what inferences an experienced naturalist could draw if she noticed a bird within her habitat of expertise. She would likely know whether or not it is unusual to find in that habitat. She would know whether it is migratory or resident. In many cases, she would be able to guess its approximate origin, and perhaps even its destination.

Now, suppose she instead discovered a bacterium. First, she must identify the organism, which is a painstaking and often expensive process. She would be somewhat lucky if the closest relative known to science could be said to be within the same genus. Depending on how she identified it, there could be considerable uncertainty even if it belonged to a previously observed species. Because of this, it is difficult to say if an observation is unusual or surprising. As for its origin, she could say nothing.

Although motion through space is conceptually simple, the movement of populations of organisms is a complex and nuanced problem. To begin, it is important to lay out the assumptions and perspectives one is prepared to make about the problem. Since we are talking about motion, we can think of this as choosing a coordinate system. The most obvious coordinate system is space; geographic distance often serves as a barrier separating populations. Among plants and animals, it is natural to conceptualize dispersal in geographic terms. A {\em species range}, as it is commonly understood by ecologists, is something that is drawn upon a map. 

Dispersal tends to structure an organism's biogeography by leaving a positive correlation between genetic distance and geographic distance. Species whose biogeography lacks this structure are said to be ``cosmopolitan''; they move through space quickly relative to their genetic divergence. There has been some recent success in observing microbial biogeography, particularly among extremeophiles. Rachel Whitaker and Thane Papke have developed a strong case for a dispersal-structured biogeography of thermophilic microbes (such as {\em Sulfolobus islandicus} and {\em Oscillatoria amphigranulata}), first by 16S rRNA phylogenetics and later using high resolution, multi-locus methods. It is only because I happened to have read Dr. Whitacker's work a few days before my visit to Mutnovsky that I was in any way prepared to think along these lines.

Nevertheless, extremeophiles are perhaps amenable to this approach because their relationship with geographic features is unusual among microbial organisms. Most microbes are found in distributions that have a less obvious relationship to geography. For example, planktonic communities in the ocean are rather undifferentiated along latitude and longitude, but differ sharply along the axes of depth and season. In soil, community composition is more strongly influenced by physical properties like pH and humidity than by geography. These phenomena are explored in {\em Part II: Oceans}.

These observations, and others like them, led Lorens Baas Becking to propose what has become the most successful conceptual model for microbial community formation : {\em everything is everywhere, but the environment selects}\footnote{{\em Alles is overal: maar het milieu selecteert.}}. This is a profound idea; it asserts that microbial dispersal is effectively infinite, and that differences of composition among microbial communities is due to selection alone. Taken literally, it means that the only difference between an Erlenmeyer flask and the entire planet is the number of unique niches. It is an assertion that scale is irrelevant.

The Baas Becking model exhorts the researcher to study the context in which microorganisms are observed. Classical microbiology is entirely built on this idea; if we apply macroscopic tools to microscopic organisms, we are perforce working through environmental selection. Most of what we know about microbial physiology, microbial diversity and community structure has been learned by careful manipulation of selection.

Nevertheless, it is important to recognize that the Baas Becking model cannot be literally true, or rather, that is only true to within the limits of a particular observational methodology. Microbes do not disperse among the continents by quantum teleportation. They must face barriers and obstacles as well as conduits and highways. Even with their rapid growth and vast numbers, this landscape of barriers and conduits must influence their propagation. Moreover, genetic divergence is a necessary and integral aspect of reproduction, and at a minimum, daughter cells cannot occupy the exact same space at the same time. Thus, biogeographic structuring must be present in all populations, but the relative scales of time and distance can vary enormously. Whether or not it can be observed is subject to limitations of methodology and technology.

Of course, Baas Becking was aware of this. In the same 1934 volume in which he proposed his eponymous hypothesis, he observed that there are some habitats that were ideally suited for one microbe or another, and yet these microbes were not present. He offered the following explanation : ``There thus are rare and less rare microbes. Perhaps there are very rare microbes, i.e., microbes whose possibility of dispersion is limited for whatever reason.''

The effects of dispersal can be inferred through the covariance of phylogenetic distance with some other measure of distance. For bacteria and archaea, the pattern of geographic verses phylogenetic distances usually does not show an obvious correlation. However, this should not be taken as evidence favoring a literal interpretation of the Bass Becking model.

To detect an unambiguous correlation between geographic distance and phylogenetic distance among lineages, not only must both measures have scales that exceed their uncertainties, but the covariance of the two measures must fall within these scales. Unfortunately, geographic distance has a ineluctable maximum scale imposed by the size of the planet, and phylogenetic distance tends to have a large minimum scale due to the uncertainties involved. Whitaker and Papke demonstrate that the covariance falls within the available scales for some organisms, but this is not generally the case.

For any given system, phylogenetic reconstruction will tend to have a practical limit on the minimum scale that can be resolved. Trees are only meaningful if they are based on features that are known to be homologous. Unambiguous identification of homologous features requires that they be conserved, and conserved features usually do not evolve rapidly (indeed, this is nearly sufficient as a {\em definition} of evolutionary conservation). Consequently, necessity biases phylogenetics in favor of long time scale variations. Homologous features suitable for phylogenetic inference on short time scales are actually far more abundant in most genomes than highly conserved features, but the ephemeral nature of these features tends to complicate their identification and limit their generality. 

At the time of writing, high throughput sequencing technology can target a $\approx$ 256 base pair subregion of the 16S gene, yielding broadly-scoped and economical surveys of bacterial and archaeal diversity. Kuo {\em et al.} estimate a substitution rate for the 16S gene ($K_{\mathrm{16S}}$) between 0.025\% to 0.091\% per million years. \cite{kuo2009inferring} At this rate of divergence, the targeted fragment of the 16S gene will accumulate one substitution per 4.3 to 15.6 million years. This fact alone goes a long way to explain why dispersal effects have not been widely observed in bacteria and archaea; any dispersal event that runs to equilibrium in less than about ten million years will be invisible to broadly-scoped sequencing strategies.

Moreover, while phylogenetic distance is often interpreted as ``time,'' there is no guarantee that phylogenetic divergence accumulates at a steady rate. It is sometimes possible to improve the correspondence of phylogenetic divergence to temporal divergence by calibrating with exogenous data, such as fossils records, archival samples or historical accounts, but there are few calibration references available for the great majority of microorganisms that do not leave mineralized structures and that are not obligate parasites or symbionts of organisms that leave fossils. Unfortunately, one cannot sidestep this problem by evoking some kind relative unit of time among a certain set of organisms. Within a single tree built using the same model of the same set of homologous features, there will exist branches that have evolved at different rates relative to other branches. There are many models that incorporate corrections for differential rates of evolution, but without calibration references, some uncertainty will always remain.

To date, substantial prior knowledge of the ecology, physiology and genetics has been required for all species of bacteria and archaea for which a dispersal-structured biogeogrpahy has been demonstrated. This prior knowledge enables the use of multi-locus sequencing of isolated cultures and the inference of trees with much finer temporal resolution than trees based on a fragment of the 16S gene. However, if we wish to observe dispersal events that are not already anticipated, and if we wish to study dispersal among uncultured organisms, we are for the foreseeable future limited by technology to low temporal resolution.

\subsection*{A change of basis}

For the time being, we are stuck with a long, poorly resolved phylogenetic axis. Our only option is to do something about the measure of distance we want to plot it against. For the covariance to fall within the usable dynamic range of phylogenetic methods, we need a measure of distance that is large relative to the phylogenetic divergence that can be resolved without ambiguity. This calls for something metaphorically similar to switching to a non-inertial frame of reference in a kinematics problem; there are many systems that are difficult to analyze from a stationary frame of reference but are straightforward in a rotating frame (high altitude ballistic trajectories, for example). In our case, we are looking for habitats that can be related to one another through a metric that varies on a scale compatible with the relatively crude metric of time that can be economically obtained from high throughput sequencing of 16S rRNA (Figure \ref{fig:intro_change_of_basis}).

\subfile{change_of_basis}

One choice that meets these criteria is to examine organisms that function as habitats for microorganisms. This is easy enough -- multicellular organisms nearly always cohabitate with microorganisms. From among these hosts, we may select groups that have been diverging long enough for usable phylogenetic signal to have accumulated among their microbial denizens. In these systems, current technology may reveal the first glimpses into a complete microbial natural history. My efforts to catch such a glimpse appear in {\em Part I: Fish}.

At the outset, my major concern was obtaining enough samples to achieve sufficient statistical power, and so I invested in some interesting tools in the hope of maximizing the practical reach of what I could achieve in the laboratory. The better of these ideas appear in {\em Part IV: Tools}. In addition to bottlenecks in the laboratory, I was also concerned that the fieldwork might also prove to be an obstacle. With that in mind, I became interested in learning how build large scale collaborations. While I do not believe it is equitable or veracious to draw a distinction between ``scientists'' and ``citizen scientists,'' the coordination of large numbers of collaborators can be a very powerful way to expand the scope of fieldwork when time and resources are limited. The single chapter that appears in {\em Part V: Space} lays out some of the results from a citizen science experiment I helped design and coordinate, and chapters two and three in {\em Part II: Oceans} are the results of a pilot project for a large citizen oceanography effort for which I designed and contributed a piece of hardware.

\printbibliography[heading=subbibliography]