\section{Abstract}

The direct ``metagenomic" sequencing of genomic material from complex assemblages of bacteria, archaea, viruses and microeukaryotes has yielded new insights into the structure of microbial communities.  For example, analysis of metagenomic data has revealed the existence of previously unknown microbial taxa whose spatial distributions are limited by environmental conditions, ecological competition, and dispersal mechanisms.  However, differences in genotypes that might lead biologists to designate two microbes as taxonomically distinct need not necessarily imply differences in ecological function.  Hence, there is a growing need for large-scale analysis of the distribution of microbial function across habitats. Here, we present a framework for investigating the biogeography of microbial function by analyzing the distribution of protein families inferred from environmental sequence data across a global collection of sites.  We map over 6,000,000 protein sequences from unassembled reads from the Global Ocean Survey dataset to $8214$ protein families, generating a protein family relative abundance matrix that describes the distribution of each protein family across sites.  We then use non-negative matrix factorization (NMF) to approximate these protein family profiles as linear combinations of a small number of ecological components. Each component has a characteristic functional profile and site profile.  Our approach identifies common functional signatures within several of the components. We use our method as a filter to estimate functional distance between sites, and find that an NMF-filtered measure of functional distance is more strongly correlated with environmental distance than a comparable PCA-filtered measure.  We also find that functional distance is more strongly correlated with environmental distance than with geographic distance, in agreement with prior studies. We identify similar protein functions in several components and suggest that functional co-occurrence across metagenomic samples could lead to future methods for de-novo functional prediction. We conclude by discussing how NMF, and other dimension reduction methods, can help enable a macroscopic functional description of marine ecosystems.
% broadly analogous to ``life-zones" in terrestrial ecosystems.

