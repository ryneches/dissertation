\section{Introduction}

Metagenomics -- large-scale sequencing of DNA isolated directly from environmental samples -- has greatly facilitated the study of microbial communities \cite{riesenfeld_metagenomics_2004, eisen_environmental_2007, wooley_primer_2010, handelsman_metagenomics_2004, rusch_sorcerer_2007, dinsdale_functional_2008}. This wealth of information has created a new set of challenges in understanding the factors underlying the functional processes mediated by microbes at community, regional and global scales \cite{eisen_environmental_2007}. For example, many variants of proteins with similar functions have been identified \cite{yooseph_sorcerer_2007}, but little is known about whether such differences have meaningful effects on ecosystem-level function. Further, genome resequencing has revealed that the genetic composition of microbes is highly variable \cite{perna_genome_2001, tettelin_genome_2005, rasko_pangenome_2008, kislyuk_genomic_2011, bates_bacterial_2011}, which suggests that information on taxonomic diversity is insufficient to characterize functional diversity.  Thus, a complementary series of analyses are necessary to quantify the functional properties of microbial communities and to explain how differences in their functional properties relate to environmental and geographic factors. Such analyses have the potential to help form the empirical foundation for the study of microbial biogeography \cite{martiny_microbial_2006, green_spatial_2006, green_microbial_2008, nemergut_global_2011}.

Data from the Global Ocean Sampling (GOS) \cite{rusch_sorcerer_2007} expedition has been previously used to investigate the relationship between microbial function and environmental variables. The GOS is appealing for such studies, since it includes samples from diverse locations and habitats, allowing investigation of the interplay among biogeography, environment, and microbial functions. The GOS data set also has important technical advantages, including: numerous samples; consistent and extensive metadata; and long, information-rich, sequence reads. Gianoulis {\em et al.} \cite{gianoulis_quantifying_2009} introduced a canonical correlation analysis (CCA) framework that was used to identify ``metabolic footprints" associated with particular environments. A follow-up study \cite{patel_analysis_2010} by the same group limited their analysis to 151 membrane protein families and used CCA again to identify relationships between protein families and environmental variables. The most recent functional analysis of GOS uses similar pathway/protein mappings and CCA methods as Gianoulis {\em et al.} \cite{gianoulis_quantifying_2009, patel_analysis_2010}, but incorporates several additional environmental measures \cite{raes_toward_2011}.  This study found that, of the climatic factors measured, temperature and sunlight were the major determinants of putative biological functions within each sample. Additionally, this study found that environmental, but not geographic, distance between samples was correlated with function. 

Eigenvector methods such as CCA and principal component analysis (PCA) are powerful tools for dimensional reduction, but pose problems for biological interpretation, because they represent data by adding and subtracting multiples of components with positive and negative elements, even when the original data has no negative entries (as with functional abundance counts).  Here we use non-negative matrix factorization (NMF) \cite{lee_learning_1999} methods to gain a complementary perspective on relationships between functions, environment, and biogeography.  Using either approach, a community can be represented as a combination of components.  NMF approximates samples using components without negative elements, and combines these components by adding positive multiples. In the context of metagenomic profiles, the components represent groups of functional or taxonomic categories that tend to co-occur in samples. Such ``parts-based'' representations have been useful for the recognition of features in human faces, text and gene expression \cite{lee_learning_1999, brunet_metagenes_2004}. In eigenvector-based decompositions, most components have positive sign for some categories and negative sign for others, and samples are also described with positive contributions from some components and negative contributions from others, preventing a straightforward parts-based interpretation.

The lower-dimensional structure identified by NMF methods is often very different from that of eigenvector-based methods. In particular, if microbial communities really are composed of fundamental components that combine in different proportions to make observed communities, NMF will use the data to approximate these underlying ``parts,'' whereas PCA will find more abstract components which have both positive and negative weights.  NMF is an efficient dimension-reduction method that has been previously used in biology, especially in identifying biologically meaningful clusters of co-expressed genes in high dimensional gene expression data sets \cite{kim_subsystem_2003, brunet_metagenes_2004, devarajan_nonnegative_2008}.  The disadvantage of NMF is that -- unlike eigenvector-based methods -- it provides only an approximate decomposition, and this decomposition is sensitive to the choice of ``rank'' -- the number of components for NMF factorization.

The starting point for analyses of microbial function biogeography is a matrix containing abundance counts of functional groups or protein families for each of the sites sampled in the study. Previous studies have focused mainly on using the KEGG \cite{kanehisa_kegg_2010} database since it contains mappings between ortholog groups (KOs) and higher level groupings (KEGG Modules/Pathways), combined with using CCA as their data reduction technique.  Here, we instead use the Pfam database \cite{finn_pfam_2010} which, in addition to full length protein families, contains many shorter protein domain families.

In this study, we make Pfam assignments for over 6,000,000 protein sequences from the GOS, resulting in 8214 unique protein families distributed across 45 sample sites. We then apply an NMF-based framework to investigate patterns of protein family distribution and their correlation with environmental variables. We illustrate how NMF can be used as an effective data reduction method and identify Pfams with common functionality in several of the NMF components. We suggest that future methods could possibly use patterns of co-occurence of protein families across metagenomic  samples as a novel non-homology based method for protein function annotation. In addition, we examine the site profiles of the components, and look for associations with geographical and environmental patterns, showing that using NMF as a filter provides further evidence that functional distance correlates better with environmental factors than geographical distance of metagenomic oceanic samples.