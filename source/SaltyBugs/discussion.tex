\section{Discussion}

We report here a workflow for the genome-wide mapping of archaeal natively expressed transcription factors using a standardized, cost-effective, high-throughput ChIP-seq platform. This is the first example of a ChIP-seq protocol for archaea. The development of this high-throughput method for mapping protein–DNA binding events in archaea should catalyze the investigation of the GRNs in this third domain of life.

While major advances have been made in mapping the GRNs of bacteria and eukaryotes, the GRNs of archaea represent a nascent area of exploration, with only a handful of genome-wide experimental studies \cite{bonneau_predictive_2007,facciotti_general_2007,kaur_coordination_2010,schmid2011two,yoon2011parallel}. Understanding regulatory mechanisms in archaea will greatly inform our understanding of the basic biology of this important domain, relevant to diverse fields of study from biogeochemistry to biotechnology. The development of improved methods for surveying archaeal GRNs is timely, coinciding with a new wave of archaeal genome sequencing that has suggested many conserved archaeal regulatory mechanisms \cite{yoon2011parallel,gelfand_prediction_2000}.

Archaea also possess an intriguing mosaic transcriptional apparatus that exhibits properties of both eukaryotic and bacterial systems. While the basal transcriptional machinery of archaea is homologous to that of eukaryotes, archaeal transcriptional regulators are more similar to those of bacteria. Studies of archaeal transcription have provided insight into both the mechanisms and evolution of information processing in the three domains of life \cite{bell_transcription_1998,geiduschek_archaeal_2005}. Likewise, deciphering archaeal GRNs holds a great potential for advancing our understanding of fundamental principles employed by GRNs across the tree of life.

The workflow presented here is cost efficient and amenable to high-throughput scaling. To increase throughput and minimize costs, we relied on recombinant strains with low-profile HA epitope-tagged target proteins and a standard anti-HA antibody for the ChIP assay. The HA epitope tag in conjunction with the well-characterized commercial antibody proved efficient for IP while minimally perturbing the target protein. The choice of the sterically slim HA epitope tag can prove quite important; we have found that the larger repeated myc epitope tag \cite{ren2000genome} can render some DNA-binding proteins nonfunctional. The HA tag does not disrupt Bat function, as seen in its ability to complement a {\em bat} knockout mutation and induce bacteriorhodopsin production ( Supplementary Figure S6 ).

A transcription factor's occupancy of possible binding sites depends, in part, on its concentration within the cell. As our ultimate goal is to map the dynamics of regulatory network rearrangement, native expression of the target transcription factor, rather than constitutive expression from a plasmid construct, was an important feature of our approach. To accomplish this, we used recombinant target proteins that were chromosomally integrated under the control of the wild-type promoter. The construction of these recombinant archaeal strains required the development of a method for generating chromosomally integrated recombinant proteins in {\em Hb.} NRC-1 , thereby expanding the genetic toolbox available for this model archaeon.

Previous ChIP-chip studies in {\em Hb.} NRC-1 used target proteins that were constitutively expressed at nonnative levels from a heterologous plasmid \cite{facciotti_general_2007, schmid2011two, schmid2009single}. The resultant protein–DNA associations are, therefore, perhaps best viewed as lists of all possible interactions rather than a snapshot of protein–DNA association network under physiological conditions. While this approach is appropriate for some applications and can offer technical advantages, such as improving ChIP efficiency for proteins present in low abundance, expression of target proteins at nonnative levels can produce artifacts in the list of protein-–DNA binding sites. In the simplest case, constitutive overexpression can drive transcription factor association to weak or nonspecific sites without significantly perturbing expression. Ambiguities concerning which binding sites are physiologically relevant can sometimes be resolved by incorporating data such as transcriptomes and regulatory motifs in the analysis. However, the perturbation of transcription factor expression can also have more serious consequences that cannot be easily resolved, such as unintended protein–protein interactions and changes in the cellular phenotype. Lastly, constitutive nonnative expression precludes investigating the dynamics of transcription factor association, a fundamental aspect in understanding the relationship between the GRN structure and function.

We demonstrated the application of our ChIP-seq protocol on two different classes of archaeal transcription factor: a general transcription factor with many binding sites (TfbD) and a more specific transcriptional activator (Bat). ChIP-seq data were analyzed with the user-friendly, open-source {\tt Pique} package, designed for identifying protein–DNA binding events in small bacterial and archaeal genomes. Our bioinformatics pipeline integrates with the Gaggle toolkit to facilitate downstream data visualization, curation and analysis.

The predicted binding sites were consistent between biological replicates and with previously published ChIP-chip results for TfbD. We observed few significant trends in the gene classes bound by TfbD, with the exception of gene functionally associated with GTPase activity ( $p$ value = $1\times 10^{−4}$ ). The lack of obvious functional partitioning of TfbD target genes is unsurprising, given this factor's broad role in global transcription initiation. Dynamic ChIP-seq experiments under different physiological conditions would likely be an appropriate future method for determining the potential regulatory roles carried out by TfbD and other archaeal general transcription factors.

The two Bat-binding sites discovered in the {\em brp} and {\em crtB1} promoters (P {\em brp}, and P {\em crtB1}) were verified by qPCR and contain two of the four previously reported occurrences of the Bat regulatory motif (P {\em bop} and P {\em blp} were not bound) \cite{baliga_genomic_2001}. It seems initially surprising that Bat binding was not detected upstream of the bacteriorhodopsin apoprotein ({\em bop}), a gene it regulates \cite{baliga_genomic_2001,gropp1994bat,leong1988characterization,baliga_coordinate_2002}. However, recent research has shown that Bat regulation of {\em bop} expression is complex and may work cooperatively with accessory proteins Brz and Brb \cite{tarasov2008small,tarasov2011small}.

Interestingly, the Bat-binding motif at P {\em brp}, and P {\em crtB1} share a single nucleotide insertion relative to the two unbound motif sites at P {\em bop} and P {\em blp}. Furthermore, the spacer sequence between the Bat motif and the TATA-box is shorter at P {\em brp} and P {\em crtB1} sites (2 and 3bp spacer, respectively), relative to the unbound P {\em bop} and P {\em blp} motif occurrences (5bp spacer) \cite{baliga_genomic_2001}. We note that these small differences in the Bat motif, in concert with our binding data, may provide some preliminary evidence to suggest a way by which Bat (and potential coregulators) distinguishes between the four predicted binding sites in a condition-specific manner.

ChIP-seq identifies protein-binding sites with fine spatial resolution and provides accurate estimates of binding site enrichment. The quantification of enrichment found at protein binding sites calculated by ChIP-seq was very similar to that determined by ChIP-qPCR (Figure \ref{SB_fig5}). Unlike ChIP-chip enrichment values, which become saturated at high levels of enrichment, ChIP-seq has excellent dynamic range, and thus provides an accurate metric for the level of enrichment at target protein-binding sites. A narrow size selection of chromatin immunoprecipitated DNA fragments enhances the enrichment in sequence coverage at target-binding sites. The use of automated size-selection instruments, such as the Pippin Prep® (Sage Science, Beverly, MA), in the preparation of ChIP-seq libraries may improve data quality.

The number of cells required for the ChIP assay was investigated to determine an optimal protocol that balances sensitivity of binding site detection with throughput and ease of sample handling. Decreasing the number of cells for the ChIP assay was found to decrease the enrichment level at target protein-binding sites, and thus the sensitivity of the assay. However, the more strongly enriched sites could still be accurately detected with $3.50 \times 10^8$ cells, equivalent to $\sim$1ml of a typical culture. The false-positive rate remains very low in the lower cell number ChIP-seq experiments. The ability to use low cell counts as input makes this approach tractable for high-throughput assessment of the more prominent binding sites in the genome, though precludes development of an exhaustive list of possible protein–DNA interactions.

By randomly subsampling deeply sequenced datasets, we determined that the required sequence coverage for the sensitive detection of binding sites corresponds to $\sim 6.5\times$ coverage of the complete genome (approximately 500K reads in {\em Hb.} NRC-1). We estimate that for the average sequencing run on the Illumina HiSeq (80 million reads) and the typical bacterial and archaeal genome ($\sim$3Mb), 130 samples can be multiplexed per lane. With this level of multiplexing, the ChIP-seq assay would cost roughly \$15 per sample. The per-sample cost is expected to drop even further with continuing improvements in the output of sequencing technologies. Relative to ChIP-chip, this ChIP-seq workflow greatly reduces the experimental cost of defining the genome-wide binding sites of target transcription factors while also improving spatial resolution. From gene tagging to data analysis, this workflow provides an excellent model for conducting large-scale, dynamic mapping of bacterial and archaeal gene regulator networks.