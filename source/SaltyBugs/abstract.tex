\section{Abstract}

Deciphering the structure of gene regulatory networks across the tree of life remains one of the major challenges in postgenomic biology. We present a novel ChIP-seq workflow for the archaea using the model organism {\em Halobacterium salinarum} sp. NRC-1 and demonstrate its application for mapping the genome-wide binding sites of natively expressed transcription factors. This end-to-end pipeline is the first protocol for ChIP-seq in archaea, with methods and tools for each stage from gene tagging to data analysis and biological discovery. Genome-wide binding sites for transcription factors with many binding sites (TfbD) are identified with sensitivity, while retaining specificity in the identification the smaller regulons (bacteriorhodopsin-activator protein). Chromosomal tagging of target proteins with a compact epitope facilitates a standardized and cost-effective workflow that is compatible with high-throughput immunoprecipitation of natively expressed transcription factors. The Pique package, an open-source bioinformatics method, is presented for identification of binding events. Relative to ChIP-Chip and qPCR, this workflow offers a robust catalog of protein-DNA binding events with improved spatial resolution and significantly decreased cost. While this study focuses on the application of ChIP-seq in {\em H. salinarum} sp. NRC-1, our workflow can also be adapted for use in other archaea and bacteria with basic genetic tools.
