\section{Materials \& Methods}

\subsection{Construction of the pRSK01 ligation-independent cloning vector for {\em Halobacterium salinarum} NRC-1}

The plasmid pNBK07 (obtained from N. Baliga, Institute for Systems Biology, Seattle, WA) has been previously used to create targeted gene knockouts \cite{kaur_coordination_2010, facciotti2010large, schmid2007anatomy, kaur_systems_2006} in the {\em Hb.} NRC-1 $\Delta${\em pyrF}  uracil auxotroph strain ({\em Hb.} NRC-1 $\Delta${\em pyrF} ). For this study, pNBK07 was modified to facilitate Gateway ligation-independent cloning of segments of DNA suitable for chromosomal modification by homologous recombination. Plasmid pNBK07 (sequence and maps in Supplementary Information ) was digested with StuI (New England Biolabs cat. \#R0187S, Ipswitch, MA), a blunt-end cutting restriction endonuclease. The digested vector was subsequently dephosphorylated with calf intestinal alkaline phosphates (New England Biolabs cat. \#M0290S) and purified via agarose gel extraction. PCR primers m13F and m13R ( Supplementary Table S1 ) were used to amplify a fragment of the pDONR221 vector containing an {\em att} P1 recombination site, the ccdB gene, a chloramphenicol resistance marker and an {\em att} P2 recombination site. This PCR product was ligated into the StuI-digested pNBK07 backbone to create the Gateway (Invitrogen, Carlsbad, CA) compatible pRSK01 vector. The pRSK01 vector was sequenced using the pNBK07F, pNBK07R, ccdB500F and ccdB500B primers. Sequences (plasmid and oligonucleotide) and plasmid maps are provided in Supplementary Information.

\subsection{Construction of chromosomally tagged transcription factors}

Chromosomally epitope-tagged transcription factors were made by way of site-specific homologous recombination in {\em Hb.} NRC-1 $\Delta${\em pyrF}. This method is analogous to that used for making in-frame gene knockouts using the pNBK07 vector as previously described ( 22 ). Two different approaches were utilized to make the epitope-tagging constructs for this study. In the first approach, PCR-mediated splicing by overlap extension (SOEing) \cite{horton1989engineering} was used to join two PCR products used to add a sequence encoding a region $\sim$500bp upstream of the bacteriorhodopsin-activator protein ({\em bat}) stop codon, an HA epitope coding sequence, a new stop codon and $\sim$ 500bp downstream of the {\em bat} genomic stop codon. PCR primers are listed in Supplementary Table S1. This PCR product was cloned into the StuI site of plasmid pNBK07, which was subsequently transformed into strain Hb. NRC-1 $\Delta${\em pyrF}. A two-step, double-crossover process was followed to chromosomally insert the HA epitope. First crossover recombinants were selected by plating on 2\% (w/v) complete media (CM) agar plates containing 20µg/ml mevinolin. Second crossover recombinants were enriched by selecting on 2\% CM agar plates containing 300µg/ml 5-fluoroorotic acid (5-FOA). The absence of a functional  {\em pyrF}  gene is required for survival on 5-FOA, indicating loss of plasmid.

The second method for chromosomal epitope tagging was used to tag the general transcription factor  {\em tfbD} . This method takes advantage of commercial DNA synthesis technologies and the new Gateway cloning compatible pRSK01 vector. In this case, a construct consisting of an attB1 recombination site, $\sim$ 500bp upstream of the  {\em tfbD}  stop codon, the sequence encoding an HA epitope tag, a stop codon, $\sim$ 500bp downstream of the  {\em tfbD}  chromosomal stop codon, and an {\em attB2} recombination site were directly synthesized by Geneart (Invitrogen, Carlsbad, CA) and delivered, cloned, in a pANY backbone vector also encoding an ampicillin-resistance marker. This vector was used directly in an {\em in vitro} recombination reaction (Gateway cloning, Invitrogen, Carlsbad, CA) with the pRSK01 vector according to manufacturer’s protocols to move the synthetic construct into pRSK01. Once the synthetic construct is inserted into pRSK01, the rest of the tagging procedure is identical to that used for pNBK07-based tagging. We have also used a combination of SOEing and Gateway recombination to directly clone PCR products, flanked by appropriate attB recombination sites, directly into pRSK01 (data not shown).

\subsection{Verification of chromosomal tagging}

The insertion of HA epitopes at the C-terminal ends of the chromosomally encoded {\em bat} and  {\em tfbD}  genes was verified both by PCR and DNA sequencing. The following PCR reactions summarized in Supplementary Figure S2 were conducted to verify insertion of the HA epitope.

The initial PCR screen (Reaction 1) verified the presence of the C-terminally tagged gene of interest in the cell using a forward primer ({\tt ct\_gene\_of\_interest \_a\_F}) located in the gene and a reverse primer complimentary to the HA epitope tag's sequence ({\tt HA\_epiope \_R}). PCR products of the expected size indicate either successful chromosomal tagging or the presence of residual tagging vector in the cell.

Recombinant strains were also screened for the presence of chromosomally encoded  {\em pyrF}  using primers flanking the chromosomally encoded gene ({\tt k\_vng167g3 \_e\_F} and {\tt k\_vng1673g \_d\_R}, Reaction 2). The presence of chromosomally encoded  {\em pyrF}  yields a 2050bp PCR product, while the disrupted  {\em pyrF}  in the {\em Hb.} NRC-1 $\Delta${\em pyrF}  strain yields a PCR product of 712bp. Reaction 2 was performed to verify that the  {\em pyrF}  gene from the plasmid has not reintegrated into the chromosome of the {\em Hb.} NRC-1 $\Delta${\em pyrF} strain.

A second  {\em pyrF}  PCR screening (Reaction 3) was carried out to confirm that the plasmid carrying the  {\em pyrF}  had been cured and that {\em pyrF} had not recombined into the chromosome of the Hb. NRC-1 $\Delta${\em pyrF}  strain. Reaction 3 amplifies a region from 465bp upstream of the {\em pyrF} stop codon to 70bp downstream of the  {\em pyrF}  stop codon (primers {\tt k\_vng1673g \_g\_F} and {\tt k\_vng1673g \_h\_R}). This final reaction yields no product in the {\em Hb.} NRC-1 $\Delta$  {\em pyrF}  strain and its derivatives. In strains that do carry the  {\em pyrF}  gene, such as wild type {\em Hb.} NRC-1, or strains transformed with either the pNBK07 or pRSK01 plasmids, a 535bp product is formed.

Finally, we screen specifically for plasmid-encoded copies of the {\em pyrF} gene using primers that amplify a segment of the  {\em pyrF}  encoded by the pNBK07 or pRSK01 vectors ( {\tt k\_vng1673g \_g\_F} and {\tt o\_pNBK07 \_a\_R}, Reaction 4). The absence of product confirms that the plasmid has been cured, when the reaction is run in conjunction with plasmid-containing positive control.

PCR products derived from PCR reaction using primers {\tt ct\_gene\_of\_interest \_a\_F} and {\tt ct\_gene\_of\_interest \_d\_R} on strains meeting all the criteria established by the verification Reactions 1–4 above were sequenced via standard Sanger sequencing to verify the integration of the HA epitope. Sequences are provided in the Supplementary Information. Tag integration was further verified by analyzing the genome re-sequencing data for each strain that was generated in the process of the ChIP-seq experiment.

\subsection{Culture preparation}

All cultures were grown in the standard CM for H. salinarum (250g/l NaCl, 20g/l MgSO 4, 2g/l KCl, 3g/L Na–Citrate, 10g/l Oxoid peptone (Oxoid cat \#LP0034) (Oxoid, Basingstoke, UK) supplemented with 50mg/l uracil and filled to volume with distilled water. Cultures were revived from −80\degree C freezer stocks and were streaked on agar plates. Starter cultures were inoculated from individual colonies and allowed to reach an optical density at 600nm of 0.7 before inoculating a culture at a starting optical density at 600nm of 0.03. Cells were grown under ambient light conditions in unbaffled flasks in a volume equal to 25\% of the flask's maximum volume. All cultures were grown at 37\degree C and shaken at 150rpm on a New Brunswick G-53 orbital shaker (New Brunswick, Edison, NJ).

\subsection{Immunoprecipitation}

Biological replicates were conducted as inoculations in separate but identical volumes on the same orbital shaker. Cells were harvested at an OD 600 between 0.9 and 1.0, which corresponds to early stationary phase for {\em Hb.} NRC-1. Cells were immediately fixed with 1\% (v/v) formaldehyde for 10min. Fixing was stopped through the addition of glycine to a final concentration of 125mM. Batches of $1.75 \times 10^{10}$ cells were removed and pelleted at 5000g. Cell pellets were washed twice with citrate-free basal salts, after which the pellets were frozen at −80\degree C. Though this was our standard input for the IP reaction, we also examined the effect of decreasing the number of input cells for the IP reaction using the {\em tfbD} ::HA strain. In addition to $1.75 \times 10^{10}$ cells, these scaling experiments also used $8.75 \times 10^9$, $3.50 \times 10^9$, $1.75 \times 10^9$ and $3.50 \times 10^8$ cells as input material for IP. The rest of the method is described in detail for $1.75 \times 10^{10}$ cells. Appropriate volumes and quantities of reagents for the scaled-down experiments are reported in Supplementary Table S3 .

Cell pellets were resuspended in 1.6ml of lysis buffer (50mM HEPES, 140mM NaCl, 1mM EDTA, 1\% (v/v) Triton X-100, 0.1\% (w/v) sodium deoxycholate, pH 7.5) containing protease inhibitors (Roche cat \#04693159001). Resuspended pellets were sonicated using a Bioruptor (Diagenode, Denville, NJ) until DNA fragment size reached an average of $\sim$500bp (2–7.5-min cycles, 30s on/30s off, high power setting).

Cell lysate was combined with 1µg of anti-HA antibody (Abcam cat \#ab9110) (Abcam, Cambridge, MA) and protein A-conjugated Dynabeads (Invitrogen cat. \#100.2D) preblocked with 5mg/ml BSA in phosphate-buffered saline and incubated overnight at 4\degree C. Dynabeads were washed two times with the lysis buffer, two times with 1ml of the lysis buffer supplemented with 500mM NaCl, two times with 1ml wash buffer (10mM Tris, 250mM LiCl, 0.5\% NP-40 (v/v), 0.5\% Na-deoxycholate (w/v), 1mM EDTA, pH 8.0) and one time with 1ml TE buffer. Enriched ChIP DNA/transcription factor complexes were eluted by the addition of 50µl elution buffer (50mM Tris, 10mM EDTA, 1\% SDS (w/v), pH 8.0) and incubation at 65\degree C for 10min. Cross-links were reversed by incubating in TE/SDS (10mM Tris, 1mM EDTA, 1\% SDS) overnight at 65\degree C. RNA was digested and DNA sample was subsequently prepared for Illumina single read sequencing.

\subsection{ChIP-seq library preparation}

Individual ChIP samples were blunt ended with T4 DNA polymerase (NEB cat. \#M0203L), Klenow large fragment (NEB cat. \#M0210L) and T4 polynucleotide kinase (NEB cat. \#M0201L) at 20\degree C for 30min. Blunt-ended DNA was 3' A tailed with 3' $\rightarrow$ 5' exo- Klenow fragment (NEB cat. \#M0212L) for 30min at 37\degree C. Adapters containing 6bp barcodes were ligated to the prepared ChIP DNA samples for 15min at room temperature with T4 DNA ligase (Enzymatics cat. \#L603-HC-L). Barcode sequences are provided Supplementary Table S4 . A background control of whole cell extract genomic DNA from each sample was prepared as mentioned above. Samples were then used as template for an 18-cycle PCR amplification. PCR products were quantified and visualized with a high-sensitivity DNA chip (Agilent cat. \#5067-4626) on a bioanalyzer (Agilent, Santa Clara, CA). ChIP and background libraries were pooled in equimolar concentrations and loaded onto a single Illumina lane.

\subsection{Western blotting}

Transcription factors were immunoprecipitated under the same conditions as ChIP methods mentioned above. IP samples were run in one dimension on 4--12\% 1.5mm polyacrylamide gel (Invitrogen cat. \#NP0335) in MOPS buffer (Invitrogen cat. \#NP0001). Protein was then blotted onto a 0.2-µm pore size PVDF membrane (Invitrogen cat. \#LC2002) at 30V for one hour in transfer buffer (25mM Tris, 192mM glycine, 10\% (v/v) methanol, pH 8.4). PVDF was blocked in 0.5\% (w/v) casein overnight and subsequently probed with HRP-conjugated anti-HA antibody (Abcam cat. \#ab1265). The blot was incubated with GE ECL plus reagents (Amersham cat. \#RPN2132) according to the manufacturer's suggestions and exposed to light-sensitive film.

\subsection{qPCR verification}

qPCR was performed on a Bio-Rad Chromo 4 Real-Time Detector (Bio-Rad, Hercules, CA) using KAPA SYBR FAST Universal 2x qPCR master mix (Kapa Biosystems cat. \#KK4601) (Kapa Biosystems, Woburn, MA) according to the supplied protocol. Primer sets for enriched regions and negative regions were designed using known enriched sites and unenriched sites, respectively, from previous ChIP-seq and ChIP-chip data (see Supplementary Table S1 for primer sequences). Fold enrichment above background was calculated as 2 to the power of cycle threshold difference between a non-enriched region and an expected enriched site. WCE extract, ChIP samples and amplified libraries were all used as template for a qPCR reaction. These were all confirmed by comparing to a set of ChIP-control reactions on the {\em Hb.} NRC-1 $\Delta${\em pyrF}  strain.

\subsection{Sequencing, processing and ChIP-seq peak calling}

Multiplexed samples were sequenced to 40bp on the Illumina GA-II. Sequences were barcode sorted and quality trimmed (minimum Phred quality 20, minimum length 25bp) using the \href{http://hannonlab.cshl.edu/fastx_toolkit/}{\tt FASTX-Toolkit} (Gordon, A and Hannon, G.J., unpublished results). Sequencing primer and adapter contamination were filtered using the {\tt TagDust} package \cite{lassmann_tagdust-program_2009}. Quality-filtered reads were mapped using {\tt Bowtie} \cite{langmead2009ultrafast} to the {\em Hb.} NRC-1 reference genome with repeat sequences masked, and {\tt SAM} format sequence files were converted to sorted {\tt BAM} files using the {\tt samtools} package \cite{li_sequence_2009}.

Putative protein–DNA binding events were detected using {\tt Pique}, a novel microbially-focused and freely available peak calling application (available at \url{https://github.com/ryneches/pique}, version tag {\tt halo\_egw}) (Neches, R.Y., Wilbanks, E.G. and Facciotti, M.T., in preparation). {\tt Pique} is written in Python and makes use of the {\tt SciPy} signal-processing subroutines \cite{scipy}. {\tt Pique} is able to operate on systems that have genomic complexities such as IS elements, gene dosage polymorphisms and accessory genomes that cause coverage variations unrelated to ChIP, or in cases where the organism under study is not identical to the reference genome. The resulting enrichment `pedestals' and `holes' can be problematic for accurately detecting binding events and calculating enrichment levels. {\em Hb.} NRC-1 has several IS elements and two plasmids that exhibit dosage variations, and so a segmented analysis was performed by providing a genomic map of these features in the reference genome.

ChIP-seq coverage data and candidate peaks were visualized using the Gaggle Genome Browser \cite{bare_integration_2010}. Shared peaks were assessed using a combination of {\tt BEDTools} \cite{quinlan2010bedtools} and custom R scripts. To assess the required sequencing depth for accurate and sensitive binding site identification, random subsampling of a 6 million read  {\em tfbD}  ChIP and WCE control runs were performed.