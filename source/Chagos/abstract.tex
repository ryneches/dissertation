\section{Abstract}

Microorganisms act both as drivers and indicators of perturbations in the marine environment. In an effort to establish baselines to predict the response of marine habitats to environmental change, here we report a broad survey of microbial diversity across the Indian Ocean, including the first microbial samples collected in the pristine lagoon of Salomon Islands, Chagos Archipelago. This was the first large-scale ecogenomic survey aboard a private yacht employing a `citizen oceanography' approach and tools and protocols easily adapted to ocean going sailboats. Our data highlighted biogeographic patterns in microbial community composition across the Indian Ocean. Samples from within the Salomon Islands lagoon contained a community which was different even from adjacent samples despite constant water exchange, driven by the dominance of the photosynthetic cyanobacterium {\em Synechococcus}. In the lagoon, {\em Synechococcus} was also responsible for driving shifts in the metatranscriptional profiles. Enrichment of transcripts related to photosynthesis and nutrient cycling indicated bottom-up controls of community structure. However a five-fold increase in viral transcripts within the lagoon during the day, suggested a concomitant top-down control by bacteriophages. Indeed, genome recruitment against {\em Synechococcus} reference genomes suggested a role of viruses in providing the ecological filter for determining the $\beta$-diversity patterns in this system.