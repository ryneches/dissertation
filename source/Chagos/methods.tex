\section{Methods}

\subsection{Sample collection}

Samples were collected as part of the Indigo V Indian Ocean Expedition 2013.\cite{lauro_common_2014} A detailed description of sampling locations and basic water characteristics is reported in Table 1. Samples were collected from approx 2m depth, into carboys that had been sanitised using household bleach (6\% v/v sodium hypochlorite) and rinsed 4 times with local water. The water was then filtered through polyethersulfone Sterivex-GP 0.22\si{\micro\meter} cartridges (catalog number SVGPL10RC, Merck Millipore) using an Athena portable peristaltic pump (Pegasus Pump Co., FL, USA) with 1.6mL of RNAlater (Life Technologies, CA, USA) and stored at -80\degree C until extraction. On average sample processing time took 45minutes between the start of the pump collection and the addition of RNAlater.

The physical characteristics of the water column were recorded using an Open Hardware Arduino-based instrument with probes for temperature, conductivity (cell constant K=1.0), pH. The instrument was calibrated using standard solutions (Atlas Scientific, NY, USA) and validated using a YSI sonde model 6600V2 (YSI Inc., OH, USA).

The following circuits were employed: Atlas Scientific EZO-pH, EDO-EC in combination with the Atlas Scientific ENV-TMP, ENV-40-pH, ENV-40-EC-1K.0 probes.

The hardware schematics and firmware can be downloaded with a BSD license at \url{https://github.com/ryneches/Atlas}.

\subsection{MODIS Satellite data and generations of GIS plots}

The AQUA/MODIS \cite{mcclain_decade_2009} spatial and temporal spectroradiometry data for the Chlorophyll concentration was plotted with monthly temporal resolutions (\url{http://neo.sci.gsfc.nasa.gov}) corresponding to each station on the Indigo V transect. Temporal and spatial estimates data were used on days with cloud cover.

The extracted values were plotted with sample positions colored by chlorophyll concentration in $\textrm{mg}/\textrm{m}^3$ on a grey canvas provided by ESRI (CA, USA) and overlaid by a map of the Longhurst provinces \cite{longhurst_estimate_1995, marineregions} using ESRI's ArcGIS 10.3 Desktop.

\subsection{DNA extraction and sequencing}

DNA was extracted from the Sterivex filters using a modified Sucrose/SDS lysis method45. Briefly, Sterivex filters were thawed at room temperature, then the RNA later was pushed out of the filter and replaced with Phosphate Buffered Saline (PBS). Filters were washed twice with PBS. Sucrose Lysis Buffer \cite{massana_vertical_1997} and lysozyme were added and the filters were incubated at 37\degree C with rocking for 30–45 minutes. 16\si{\micro\meter} of RNase Cocktail (Life Technologies, CA, USA), 40\si{\micro\meter} Proteinase K (20mg/ml, Qiagen, CA, USA) and 40\si{\micro\meter} 10\% SDS were added to the filter and incubated at 55\degree C for 1–2 hours. At that point the lysate was extracted with a standard Phenol-Chloroform-Isoamyl alcohol extraction and precipitated with 100\% ethanol. Pelleted DNA was washed with ethanol (70\% v/v) and resuspended in nuclease free water.

Amplicon libraries were generated by following Illumina's 16S Metagenomic Sequencing Library Preparation Protocol, using 12.5ng of template DNA per reaction. PCR cycles for the first PCR were reduced to 20 to avoid PCR biases from over-amplification. The following primers were used for the initial amplification, consisting of an Illumina-specific overhang sequence and a locus-specific sequence:

{\tt 926wF\_Illum: 5'-TCGTCGGCAGCGTCAGATGTGTATAAGAGACAGAAACTYAAAKGAATTGRCGG}

{\tt 1392R\_Illum: 5'-GTCTCGTGGGCTCGGAGATGTGTATAAGAGACAGACGGGCGGTGTGTRC}

\noindent This universal primer pair targets the V6-V8 hyper-variable regions of the 16S/18S Ribosomal RNA gene and has been shown to capture, in a single reaction, the microbial diversity of Archaea, Bacteria and Eukarya. \cite{wilkins_advection_2013}

PCR reactions were then purified with Agencourt AMpure XP beads (Beckman Coulter) and subjected to a second round of amplification for library barcoding according to Illumina's recommendation.

Library quantitation was performed using Invitrogen's Picogreen assay and the average library size was determined by running the libraries on a Bioanalyzer DNA 1000 chip (Agilent). Library concentration was normalised to 4nM and the concentration was validated by qPCR on a ViiA-7 real-time thermocycler (Applied Biosystems), using qPCR primers recommended in Illumina's qPCR protocol and Illumina's PhiX control library as standard. The libraries were then pooled at equal volumes and sequenced in two lanes of an Illumina MiSeq V3 run at a final concentration of 4 pM and a read-length of 301bp paired-end. The SSU amplicon data was deposited at NCBI under BioProject PRJNA281973.

\subsection{RNA extraction and sequencing}

RNA was extracted using the Totally RNA kit (Life Technologies, CA, USA) with protocol modifications. After removing all residual RNAlater, the sterivex filter was washed with 25mL of RNase free PBS, followed by the addition of 1.6mL denaturation solution from totally RNA kit (Life Technologies, CA, USA). The filter was incubated at room temperature on a rotating platform for 30 minutes, inverted and incubated for a further 30 minutes. The denaturation solution was withdrawn, divided into two microfuge tubes and the remaining RNA extraction was completed as described in the Totally RNA manual (Life Technologies, CA, USA) following the phenol/chlorofom/isoamyl alcohol protocol. In-vitro ribosomal RNA depletion was not performed allowing for the analysis of the activity of different taxonomic groups using RiboTagger (see below).

Prior to library preparation, the quality of the RNA samples was determined by running the samples on a Bioanalyzer RNA 6000 Nano Chip (Agilent). Sample quantitation was performed using Invitrogen’s Ribogreen assay. To rule out DNA contamination, the RNA samples were also subjected to Invitrogen's Picogreen assay.

Next-generation sequencing library preparation was performed by following Illumina's TruSeq Stranded Total RNA protocol with the following modifications: the mRNA purification step was omitted and instead, 400ng of total RNA were directly added to the elute-fragment-prime step. The PCR amplification step, which selectively enriches for library fragments that have adapters ligated on both ends was performed according to the manufacturer's recommendation, but the number of amplification cycles was reduced to 12. Each library was uniquely tagged with one of Illumina's TruSeq LT RNA barcodes to allow library pooling for sequencing.

\subsection{Library quantitation was performed using Invitrogen}

Analysis of microbial community composition and diversity
Illumina amplicons were analysed using the {\tt QIIME} pipeline \cite{caporaso_qiime_2010} with standard protocols. Briefly, paired end sequences were joined using {\tt FastqJoin} (\url{http://code.google.com/p/ea-utils}) and libraries were de-mutiplexed and quality filtered to truncate reads at positions with Phred scores $<\textrm{Q}20$ retaining only reads $>75$bp and with $<3$ low quality bases and no N characters. Chimeras were detected and removed using {\tt USEARCH} \cite{edgar_search_2010} and {\em de novo} OTUs defined at 97\% sequence similarity using {\tt UCLUST}. \cite{edgar_search_2010} Taxonomy was assigned against the SILVA Database (v.119) \cite{quast_silva_2013} using the Mothur taxonomy assigner49. Representative sequences from each OTU were aligned with the SILVA database using {\tt PyNAST} \cite{massana_vertical_1997} and a phylogenetic tree constructed using {\tt FastTree} 2. \cite{price_fasttree_2010} This tree was used to measure the phylogenetic similarity between samples using weighted UniFrac. \cite{lozupone_unifrac:_2005, lozupone_unifrac:_2011} The representative OTU sequences were also further analysed against the curated Protist Ribosomal Reference database \cite{guillou_protist_2013} and the taxonomy of eukaryotes was assigned using a {\tt UCLUST} \cite{edgar_search_2010} consensus assignments. Picogreen assay and the average library size was determined by running the libraries on a Bioanalyzer DNA 1000 chip (Agilent). Library concentration was normalised to 2nM and the concentration was validated by qPCR on a ViiA-7 real-time thermocycler (Applied Biosystems), using qPCR primers recommended in Illumina's qPCR protocol and Illumina's PhiX control library as standard. The libraries were then pooled at equal volumes and sequenced in two lanes of an Illumina HiSeq2500 V1 rapid run at a final concentration of 10pM and a read-length of 101bp paired-end. RNA samples were sequenced on the Illumina HiSeq2500 with Rapid V1 chemistry. The raw metatranscriptomic datasets were deposited at NCBI under BioProject {\tt PRJNA281973}.

To ensure even sampling depth for subsequent analyses, OTU abundance data was rarefied to the lowest number of sequences for a sample (6871 sequences). Subsequent rarefactions at higher cut-offs did not significantly alter patterns in $\upbeta$-diversity or community composition (results not shown). Taxa abundance data derived from this table and the matrices of Bray-Curtis and UniFrac dissimilarity were visualised using PRIMER V6 \cite{clarke2006primer} utilising Multidimensional Scaling (MDS) and SIMPER analysis. \cite{clarke_nonparametric_1993} $\upalpha$ - and $\upbeta$-diversity metrics were calculated using {\tt QIIME}.

\subsection{RiboTagger analysis of transcriptome data}

Total RNA reads were analysed by {\tt RiboTagger} (manuscript in preparation. Version 0.7.0. Available at \url{https://github.com/xiechaos/ribotagger}). Briefly, a position specific scoring matrix (PSSM) was used to scan all sequencing reads in order to detect sequences immediately next to the V4 region of 16S rRNA, and report a 33nt tag sequence of the V4 region downstream of the PSSM matched position. Frequencies of each different V4 tags were normalised by the total sequencing reads covering a V4 tag. Multidimensional Scaling was then carried out on the V4 tag frequency data with Bray-Curtis dissimilarity using PRIMER V6. \cite{clarke2006primer}

\subsection{In-silico rRNA depletion and transcriptome analysis}

Total RNA reads were screened by {\tt dc-megablast} against SILVA rRNA database (v. 115) \cite{quast_silva_2013} to remove potential rRNA reads. The remaining reads were used to search against the NCBI non-redundant dataset using DIAMOND. \cite{buchfink_fast_2015} A Lowest Common Ancestor (LCA) algorithm, implemented in {\tt LCAmapper} as part of {\tt mtools} (\url{http://ab.inf.uni-tuebingen.de/data/software/megan5/download/mtools.zip}) was used to report functional composition of each dataset based on DIAMOND search results, referencing reads to KEGG Orthology (KO) identifiers. \cite{kanehisa_kegg:_2000} Full results are reported in Supplementary Table 5.

We then generated read count matrices with orthologous genes index in rows and samples in columns, and further normalised these using the variance stabilising transformations implemented in DESeq. \cite{anders2010differential} We further aggregated orthologous gene data to pathways as defined by KEGG, and computed the average of per-KO normalised read count as an overall summary of pathway expression. Full results are reported in Supplementary Tables 6 and 7.

The in-silico rRNA-depleted metatranscriptomes were also searched against the POG database \cite{kristensen_orthologous_2013} using {\tt RAPsearch2}. \cite{zhao_rapsearch2:_2012} Statistically significant differences between the number of hits to each POG in a comparison of the inside day samples (SAI01-02) and outside samples (SAO01-02) were evaluated with the resampling approach of Rodriguez-Brito \cite{rodriguez-brito_application_2006} using 1,000 subsamples of 10,000 POG hits each at a significance threshold of 98\%. Amongst the differentially abundant POGs, only those that had at least 1,000 counts across all samples and a viral quotient higher than 0.8 were considered for further analysis. The viral quotient (VQ) is a measure of the likelihood of finding a POG only in phage genomes \cite{kristensen_orthologous_2013} and is calculated as the ratio:

\begin{equation}
VQ = \frac{v}{v+h}
\end{equation}

\noindent where $v$=number of hits to viral genomes in NCBI/total number of viral genomes in NCBI and $h$=number of hits to prokaryotic genomes in NCBI/total number of prokaryotic genomes in NCBI (with hits to predicted prohage regions excluded). Using viral quotient and minimum sample size cutoffs increased the confidence that the observed changes in POG hits were `true' proxies for viral infection.

\subsection{Recruitment of RNAseq reads to representative genomes of {\em Synechococcus}}

The relative proportion of different {\em Synechococcus} genotypes within the mRNA reads was estimated with {\tt blastn} \cite{schloss_introducing_2009} against 18 available marine Synechococcus genomes using default parameters except for e-value of $10^{-5}$ and a single best hit was reported for each read. Recruitment of mRNA reads against the WH8109 genome was performed with {\tt Bowtie2} \cite{langmead_fast_2012} and the sam alignment was converted to a graph using BRIG \cite{alikhan_blast_2011} and the coverage plots were produced in {\tt R}.
