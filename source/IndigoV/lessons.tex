\section{Lessons Learned from the Indigo V Indian Ocean Expedition}

Many projects have struggled with the unique technical, logistical, organizational, and ethical issues that arise for each discipline when researchers endeavor to involve citizen scientists. In 2013, the Indigo V Indian Ocean Expedition was conceived as a pilot project and learning laboratory for citizen science approaches to oceanography. The team sailed S/Y {\em Indigo V} -- a 61-foot Nautor Swan sailing yacht -- across the Indian Ocean from Cape Town, South Africa, to Phuket, Thailand. The three legs of the journey covered approximately 5,800 nautical miles. During this expedition, instruments and methods adaptable to citizen scientist deployment were tested aboard small vessels not designed or equipped for research. In all but the heaviest seas, the crew was able to inventory the surface water population of bacterioplankton using a simple pump and filtration apparatus and make basic measurements of ocean physics and chemistry. DNA and RNA were successfully recovered from samples preserved using a nontoxic salt solution (RNAlater, Qiagen, Valencia, California).

The prototype ocean sampling microbial observatory (OSMO) is currently being ruggedized and automated for citizen-science-based collections of bacterioplankton samples. This device is being developed as a collaboration between the {\em Indigo V} team members in their laboratories in the US, Singapore, and Australia and will autonomously sample microbial populations onto filters and preserve them. The sailor/scientist would be responsible for metadata collection, uploading that data to a central database, and shipping the samples back to the lab for processing. The total cost of design, prototyping, field-testing, and commercializing this device is less than US\$200,000. Ultimately, the total cost of microbial sample collection, processing, and sequencing using this approach and device could be reduced to $\approx$US\$1,500 per sample or less. It will be cost-effective to inventory the microbial community of an ocean basin. The citizen science can be extended to the data analysis phase by online annotation tutorials.

Many aspects of science aboard a sailing yacht are similar to science aboard a research vessel. There is an intense focus on collecting samples; by simplifying the sample collection methodology, this task can be taught to sailors. Similarly, observations aboard a sailing yacht are a part of daily life-situational awareness is essential for the safety of the crew and the boat. Wind speed, sea state, sea height, and currents are observational data that can be collected by sailors to improve global ocean models (or verify predictions).

By employing sail power, the Indigo V project demonstrated that an entire four-month expedition, sampling a wide range of waters with a variety of instruments, costs the equivalent of a day or two of ship time aboard an oceanographic research vessel. Relative to a typical research vessel, the use of sail power reduced carbon emissions resulting from vessel operations by approximately 1000-fold. For comparison, a recent global oceanographic project (\url{http://www.expedicionmalaspina.es/}) collected samples at 180 stations with a budget of US\$23 million \cite{malaspina, upv-ehu}, and another \url{http://oceans.taraexpeditions.org/} sampled 375 stations with a budget of over US\$12 million \cite{tara-embl, tara-2009, tara-2010, tara-2011}. The Indigo V Indian Ocean Expedition collected samples at 50 stations for less than US\$75,000. Imagine what the thousands of yachts that are already out on the water could do.
