\section{Introduction}

We live on a vast, underexplored planet that is largely ocean. Despite modern technology, Global Positioning System (GPS) navigation, and advanced engineering of ocean vessels, the ocean is unforgiving, especially in rough weather. Coastal ocean navigation, with risks of running aground and inconsistent weather and sea patterns, can also be challenging and hazardous. In 2012, more than 100 international incidents of ships sinking, foundering, grounding, or being lost at sea were reported \url{http://en.wikipedia.org/wiki/List_of_shipwrecks_in_2012}. Even a modern jetliner can disappear in the ocean with little or no trace \cite{mcnutt_hunt_2014}, and the current costs and uncertainty associated with search and rescue make the prospects of finding an object in the middle of the ocean daunting \cite{allen_us_2010}.

Notwithstanding satellite constellations, autonomous vehicles, and more than 300 research vessels worldwide \url{http://www.wikipedia.org/wiki/List_of_research_vessels_by_country}, we lack fundamental data relating to our oceans. These missing data hamper our ability to make basic predictions about ocean weather, narrow the trajectories of floating objects, or estimate the impact of ocean acidification and other physical, biological, and chemical characteristics of the world's oceans. To cope with this problem, scientists make probabilistic inferences by synthesizing models with incomplete data. Probabilistic modeling works well for certain questions of interest to the scientific community, but it is difficult to extract unambiguous policy recommendations from this approach. The models can answer important questions about trends and tendencies among large numbers of events but often cannot offer much insight into specific events. For example, probabilistic models can tell us with some precision the extent to which storm activity will be intensified by global climate change but cannot yet attribute the severity of a particular storm to climate change. Probabilistic modeling can provide important insights into the global traffic patterns of floating debris but is not of much help to search-and-rescue personnel struggling to learn the likely trajectory of a particular piece of debris left by a wreck.

Oceanographic data are incomplete because it is financially and logistically impractical to sample everywhere. Scientists typically sample over time, floating with the currents and observing their temporal evolution (the Langrangian approach), or they sample across space to cover a gradient of conditions—such as temperature or nutrients (the Eulerian approach). These observational paradigms have various strengths and weaknesses, but their fundamental weakness is cost. A modern ocean research vessel typically costs more than US\$30,000 per day to operate—excluding the full cost of scientists, engineers, and the cost of the research itself. Even an aggressive expansion of oceanographic research budgets would not do much to improve the precision of our probabilistic models, let alone to quickly and more accurately locate missing objects in the huge, moving, three-dimensional seascape. Emerging autonomous technologies such as underwater gliders and {\em in situ} biological samplers (e.g., environmental sample processors) help fill gaps but are cost prohibitive to scale up. Similarly, drifters (e.g., the highly successful Argo floats program) have proven very useful for better defining currents, but unless retrieved after their operational lifetime, they become floating trash, adding to a growing problem.

Long-term sampling efforts such as the continuous plankton recorder in the North Sea and North Atlantic \cite{warner_sampling_1994} provide valuable data on decadal trends and leveraged English Channel ferries to accomplish much of the sampling. Modernizing and expanding this approach is a goal of citizen science initiatives. How do we leverage cost-effective technologies and economies of scale given shrinking federal research budgets?