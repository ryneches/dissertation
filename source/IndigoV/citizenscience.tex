\section{Citizen Scientists Can Make an Important Contribution}

There are many actions that can be taken to improve the precision of our models, but the most obvious is to increase spatial and temporal density of our observations. However, the cost of oceanographic research vessels makes this impractical. The inevitable conclusion is that observations must be obtained by some other means. We propose a worldwide effort to empower sailors and retrofit sailboats to increase coverage of sample and data collection along common routes around the world.

Modern oceanographic research vessels are large and expensive because they are designed to be general-purpose scientific platforms. They are sophisticated laboratory facilities that serve the diverse needs of the scientific community for many decades. These vessels are costly because their scientific capabilities are both wide ranging and deep penetrating. The ocean is too vast for any vessel to see very much of it, no matter its capabilities. Maximizing the number of observers, rather than the capabilities of observers, requires a very different approach to the choice of vessel, personnel, instrumentation, and protocol.

Can meaningful data be collected with the kind of narrowly focused, low-cost instrumentation that is easily mass produced and deployable? If so, what vessels will carry it, and what personnel will operate it? Many aspects of modern oceanography, such as locating an underwater object, require sophisticated equipment and trained experts. However, some of the most important types of observations require only that one be in the right place at the right time with simple instrumentation or sampling equipment. Important data can be gathered by anyone who can follow basic instructions. This is the premise of ``citizen science'' (Box 1). Rather than dispatching scientists into the environment to collect data, scientists may instead train people who already interact with the environment to apply the scientific method to phenomena they already observe. With or without an invitation, citizen scientists exist. There is an urgent need to make a place for them in the scientific community.
