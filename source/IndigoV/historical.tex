\section{Historical Perspective}

A mistaken and modern perception is that science is an elitist profession, relegated to well-funded laboratories with complex instrumentation run by professors with years of advanced education. Historically, this was not always the case. In fact, people who conducted scientific research as a hobby achieved some of the greatest discoveries in history. For example, Leonardo da Vinci painted portraits for income while doing science in his spare time. Gregor Mendel, an Augustinian friar, discovered the basis of genetic inheritance while working in the garden of his monastery, and Michael Faraday laid the foundation of electromagnetic induction while working as an apprentice bookbinder and bookseller, educating himself. Arguably the most famous “citizen oceanographer” was Charles Darwin, who had no formal training in biology but became one of the most celebrated and influential evolutionary biologists in history. He detailed the geology and formation of coral reefs during the 1832–1836 voyage of HMS Beagle. Similarly, Benjamin Franklin, also a “citizen oceanographer,” published a number of ideas on Atlantic Ocean currents, catamaran hulls, and sea anchors and designed a spill-proof bowl for eating soup on board a ship in stormy seas.

Despite this long tradition, the involvement of amateurs in oceanographic discoveries declined in the 20th century, perhaps contributing to the growing misunderstanding of scientific jargon by the public when it pertains to ocean and atmospheric circulation. This has led to the current political shape-shifting of scientific results (e.g., the climate change debate). In the last few years, mostly due to technological breakthroughs, we have witnessed revitalization in the participation of civilians in data collection: Alan Irwin described the social aspects of this revolution in 1995 \cite{irwin1995citizen} and coined the term ``citizen science."

The United States National Weather Service Cooperative Observer Program (NWS-COOP) is a great example of a successful citizen science initiative. The program was established in 1890 and utilizes a network of more than 11,000 volunteers to provide observational data of basic weather parameters. Similarly, the USA National Phenology Network is a group of scientists and trained citizen scientists who collect observations about plant and animal phenology. This multidecadal program helps scientists, for example, better understand climate change \cite{ault2013false}.

With thoughtfully designed and well-tested equipment and protocols, citizen scientists can gather vast quantities of oceanic data or samples for analysis (Box 2). Three technologies have provided the technical means for networked data collection: the miniaturization of sample collection devices, the progressive reduction in the cost of sequencing, and the computing and easy data sharing of cloud-based analysis \cite{haklay2013citizen}.

Oceanography using large ships has a significant carbon and economic footprint, but spatially extensive and temporally intensive data are needed. More specifically, in marine microbiology, data collection has become the bottleneck, since it is currently impossible to quantify the totality of oceanic microbial communities and their environmental drivers by remote sensing or individual research cruises. Expansive budgetary cuts to environmental sciences around the globe and the concurrent need to renew an aging fleet of ocean vessels \cite{cressey2014us} underscore the urgency.