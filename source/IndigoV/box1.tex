\section{Box 1. Citizen Science Primer}

Citizen science is a manner of collecting data and observations in which collaborators who may lack credentials and formal institutional affiliation can contribute to the work. Because one does not vet collaborators on the basis of affiliation and credentials, a citizen science research project must be specific and self-contained in terms of what is asked of collaborators.

For example, rather than requiring a master's degree in entomology, a citizen science project might ask if a candidate can learn to identify a particular species of ant using a dichotomous key.