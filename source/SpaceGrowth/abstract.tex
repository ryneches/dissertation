\section{Abstract}

\textbf{Background.} While significant attention has been paid to the potential risk of pathogenic microbes aboard crewed spacecraft, the non-pathogenic microbes in these habitats have received less consideration. Preliminary work has demonstrated that the interior of the International Space Station (ISS) has a microbial community resembling those of built environments on Earth. Here we report the results of sending 48 bacterial strains, collected from built environments on Earth, for a growth experiment on the ISS. This project was a component of Project MERCCURI (Microbial Ecology Research Combining Citizen and University Researchers on ISS).\\
\textbf{Results.} Of the 48 strains sent to the ISS, 45 of them showed similar growth in space and on Earth using a relative growth measurement adapted for microgravity. The vast majority of species tested in this experiment have also been found in culture-independent surveys of the ISS. Only one bacterial strain showed significantly different growth in space. \textit{Bacillus safensis} JPL-MERTA-8-2 grew 60\% better in space than on Earth.\\
\textbf{Conclusions.} The majority of bacteria tested were not affected by conditions aboard the ISS in this experiment (e.g., microgravity, cosmic radiation). Further work on \textit{Bacillus safensis} could lead to interesting insights on why this strain grew so much better in space.