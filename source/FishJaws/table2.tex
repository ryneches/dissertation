\begin{table}[]
\centering
\begin{tabular}{@{}llll@{}}
\toprule
Species & N & Location & SL range(mm) \\ \midrule
\textit{\textbf{Lates niloticus}} & \textbf{7} & \textbf{Victoria Basin} & \textbf{109-168} \\
\textbf{\textit{Harpagochromis} `orange rock hunter'} & \textbf{5} & \textbf{Victoria Basin} & \textbf{98-135} \\
\textit{\textbf{Pyxichromis orthostoma}} & \textbf{3} & \textbf{Victoria Basin} & \textbf{102-109} \\
\textit{\textbf{Harpagochromis cf serranus}} & \textbf{3} & \textbf{Victoria Basin} & \textbf{106-113} \\
\textbf{\textit{Harpagochromis} `two stripe white lip'} & \textbf{8} & \textbf{Victoria Basin} & \textbf{108-164} \\
\textit{Lipochromis} `matumbi hunter' & 5 & Victoria Basin & 91-102 \\
\textit{Lipochromis parvidens} & 5 & Victoria Basin & 115-121 \\
\textit{Champsochromis caeruleus} & 2 & Lake Malawi & 178-193 \\
\textit{Nimbochromis fuscotaeniatus}* & 1 & Lake Malawi & 123 \\
\textit{Nimbochromis venustus}* & 1 & Lake Malawi & 149 \\
\textit{Rhamphochromis longiceps} & 3 & Lake Malawi & 125-196 \\
\textit{Boulengerochromis microlepis} & 4 & Lake Tanganyika & 172-193 \\
\textit{Bathybates minor} & 1 & Lake Tanganyika & 84 \\
\textit{Cyphotilapia frontosa} & 2 & Lake Tanganyika & 85-129 \\
\textit{Cichla ocellaris} & 4 & South America & 93-100 \\
\textit{Parachromis dovii}* & 1 & Central America & 179 \\
\textit{Parachromis managuense}* & 2 & Central America & 111-165 \\
\textit{Petenia splendida} & 4 & Central America & 94-109 \\ \bottomrule
\end{tabular}
\caption{Predators used in pharyngeal probe experiments. Bolded species were used in prey processing trials. Starred species were averaged together by genus (see Figure \ref{FJ_fig2}a).}
\label{FJ_table2}
\end{table}