\section{Abstract}

As a single lineage evolves and branches into a group of distinct new lineages, it is said to diversify. However, it is rarely the case that this process occurs in isolation; evolving lineages interact with one another with varying intensity over time. When the histories of interacting lineages are inferred, their phylogenies resemble ``tangled trees.'' Most existing methods for studying these systems fall into two categories; either they test for codiversification by fitting data to an idealized model, or in cases where codiversification is known to have occurred, they endeavour to predict the most likely history. Here, we present a comparative approach to codiversification. Using the graph spectra of networks of interacting organisms, we construct a similarity metric that permits the use of established clustering and machine learning techniques to classify interactions that exhibit similar phylogenetic and ecological structures. Using this method, we are able to propose roles for unidentified host-associated bacterial clades based on the similarity of their interactions with their hosts to ecological interactions where the roles are known.