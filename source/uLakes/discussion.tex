\section{Discussion}

We compared patterns of genomewide differentiation between three sympatric and reciprocally monophyletic species pairs of haplochromine cichlids each involving a small insect picker and a large piscivore, ecomorphs that are archetypal of the replicated adaptive radiations in different lakes. We reveal strong differences among our species pairs in whole-genome estimates of fixed and absolute divergence, likely associated with time and gene flow. The pairs also contain a modest signal of parallel evolution across both old and young pairs. We discuss each of these results in turn.

Our mitogenome phylogeny corroborates the previous phylogenetic hypotheses for these species \cite{wagner2012ecological}. The young divergence time between the two Lake Victoria cichlids is consistent with speciation and ecological differentiation within the lake itself, as well as possible ongoing gene flow. Direct gene flow between these species is extremely unlikely based on the extreme amount of ecological and phenotypic divergence, but indirect gene flow via other species is a possibility. The much older divergence between the Mweru lineages is consistent with an old split between the southern African serranochromine cichlids and the Mweru-Orthochromis, two old haplochromine lineages \cite{schwarzer2012repeated, wagner2012ecological}. Our mitogenome data does not cleanly resolve differing ages for our Victoria and Kivu pairs, although patterns of fixed divergence across the genome suggest that the Mweru pair is older.

We recover strongly divergent patterns in the level of fixed differences ($d_f$) observed in each lake pair. Victoria, the youngest of the three radiations, exhibits the lowest level of $d_f$, which most 10-kb blocks containing no fixed differences between the fish-eating {\em Harpagochromis} sp. `checkmate' and the insect-eating {\em Paralabidochromis flavus}. Lake Kivu exhibits significantly higher levels of $d_f$ than Victoria, suggesting slightly older divergence within the Kivu pair. Our young pairs exhibit fixed divergence at only 42,849 sites in Victoria and 44,531 sites in Kivu of a total of 392 million sites examined; the true fraction will be even lower. In contrast, $d_f$ is strongly elevated in the two ecologically and phenotypically similarly divergent but ancient haplochromine species of Lake Mweru, with over 2 million fixed divergent SNPs. Average $d_f$ levels are over forty times higher in Mweru as compared to Kivu or Victoria, suggesting that {\em Orthochromis} and {\em Serranochromis} have been strongly isolated for a long time, congruent with our divergent mitogenome data for these species.

The distribution of $d_f$ is quite different in Victoria and Kivu as compared to Mweru (Figure \ref{UL_fig2}C). Victoria and Kivu possess $d_f$ distributions that resemble exponential functions, with the highest levels of divergence strongly concentrated in only a few 10-kb blocks with divergence levels twenty to fifty times higher than the rest of the genome. The small, infrequent nature of these $d_f$ outliers, spread out widely across the genome, confirms results from a previous study \cite{brawand2014genomic} that did not locate any larger islands of differentiation between Lake Victoria species and very little fixation with RADseq markers. The $d_f$ distribution of Mweru is similar but elevated, with the exception of a small number of blocks with little to no divergence. The overall elevated pattern of divergence in Mweru relative to Kivu and Victoria does suggest a long period of genomewide reproductive isolation between the two Mweru lineages.

In Kivu and Victoria, less than four per cent of the genomes exhibits $d_f$ elevated above the level of the mean plus two standard deviations, but this is true for just over two per cent of the genomes in Lake Mweru. These results support our prediction that genomic divergence should be more homogenous in comparisons between fully reproductively isolated species, with potential `islands' beginning to submerge in a rising tide of neutral divergence. Comparisons of outlier regions in young pairs to genomewide divergence in Mweru also support this pattern, as all by the very largest $d_f$ outlier blocks in the young pairs are `submerged' below the average level of genomewide divergence in Mweru.

We reveal elevated $D_{XY}$ values in $d_f$ outlier regions across all three species comparisons. Genomic comparisons of closely related species in other taxa revealed that regions elevated for measures of relative divergence like $d_f$ and FST generally did not also exhibit high $D_{XY}$ values relative to the genomewide average \cite{cruickshank2014reanalysis}. Such discrepancy between relative and absolute measures of divergence have led to debate about the validity of the interpretation of heterogeneous genomic divergence using relative divergence measures as evidence for speciation with gene flow. Islands of relative but not absolute divergence can also be attributed to effects of background selection in already completely isolated species or geographically isolated populations \cite{cruickshank2014reanalysis}.

Why do we find such strong congruence between relative and absolute measures of divergence in our case? One possibility is that our comparisons involve far more and stronger ecological and functional divergence than is typical for other published species contrasts. With a few exceptions \cite{estrada2002patterns, lehmann2008molecular}, many systems tested for `genomic islands' exhibit no known ecological divergence between sister species at all, let alone the combined body size, trophic level, feeding mode, habitat specialization, social behaviour and male coloration pattern divergence we observe in each of our cichlid comparisons. It was with this in mind that from each lake, we chose species belonging to very different ecomorphs. A second contributing factor could be that all our species pairs live in sympatry and, the pairs in Lakes Victoria and Kivu can hybridize. Under such conditions, background selection is unlikely to generate spurious islands of relative divergence. We suggest that future studies of examining `genomic islands' in the context of speciation with gene flow should first establish that ecological and functional differences actually exist in the taxa of interest, as well as sampling them from real sympatry.

It is also possible that our limited sample size of one genome per species drives some of the correlation between $d_f$ and $D_{XY}$ we observe here, as well as the relatively large block sizes we use. However, we suspect that reducing block size or increasing the number of genomes per species will merely reduce variance in genomewide divergence of our young pairs. For instance, if we reduced our analysis window from 10 kb to 1 kb, a 10-kb block with one fixed divergent SNP, one of the most common types of block in our analysis of Kivu and Victoria, would be reduced to nine blocks with 0 SNPs and a block with one SNP, reducing the variance in $d_f$ levels across the genome. This reduction in genomewide variance in $d_f$ would be more likely to make the few regions with high $d_f$ and $D_{XY}$ easier to differentiate from the rest of the genome. However, in the older Mweru pair, which contains many more SNPs per block, it is possible that adding more samples or reducing block size would have the potential to reduce the connection between high $d_f$ and $D_{XY}$ values.

We recover only very modest evidence for genomic parallelism associated with the replicated divergence into a small insect picker and a larger piscivore in three different lakes. Outlier regions predicted for $d_f$ in one lake also exhibit elevated $D_{XY}$ in the other two lakes. This pattern holds true for the top 1\% and 0.1\%, but not the top 0.01\% outliers, which comprise fewer than five $d_f$ blocks in each lake. These top loci do not show strong correspondence between lakes, suggesting that either our top outliers are not associated with the specific traits that characterize our replicated ecomorph divergences, or that they represent the most strongly divergent out of a larger set of loci that can be recruited for divergence in these traits (i.e. polygenic adaptation), but a different subset is most strongly recruited in each case, and the power to detect increases in $D_{XY}$ decreases with the number of loci retained in the analysis. If some of these parallel changes are due to recent sweeps, $D_{XY}$ might actually be reduced in $d_f$ outlier regions identified in other pairs, hindering our ability to detect a signature of parallelism across lakes.

It is also possible that the most strongly selected loci affect different traits in each of our species pairs. While extremely similar, our pairs are not identical to each other. All three insectivore species exhibit differences in male coloration and body depth relative to each other, and our {\em Orthochromis} species lacks the eggspots on the anal fin that all the other species have. Our predators are much more similar to each other, although the {\em Serranochromis} species exhibits more eggspots in its anal fin than the two {\em Harpagochromis} from Kivu and Victoria. Still, our results do suggest that ecological divergence across these lakes likely involves polygenic adaptation.

One possibility is that the observed patterns of genomic divergence are primarily due to body size divergence within our pairs. Our three lake pairs all have extreme divergence in body size, with the piscivore species attaining sizes over twice that of the insect pickers (Figure \ref{UL_fig1}). It is highly likely that this divergence is the result of natural selection, as a larger body size often enables more effective foraging on other fishes \cite{wainwright1995predicting}. Body size divergence is known to be a highly polygenic trait in other organisms \cite{turner2011population}, whereas traits associated with functionally important variation in specific bones and muscles often map to only a few quantitative trait loci \cite{arnegard2014genetics}. As such, the patterns of genomic divergence we see across all pairs may involve many small effect loci associated with body size, although more work will be required to rigorously assess the genetic basis of body size divergence in cichlids.

We observe a clear signature of genomic divergence and parallel evolution in our comparison of functionally and ecologically divergent cichlids from three lakes with very different evolutionary histories and divergence times. We suggest that analyses of whole-genome divergence can be most helpful for understanding the genomic signature of adaptation and speciation when the sampling design is well informed by natural history.