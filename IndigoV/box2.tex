\section{Box 2. Citizen Scientists Could Collect Several Types of Important Oceanographic Data}

\begin{itemize}

\item{\em Biological samples.} Our ability to monitor the status of the world's oceans and evaluate the effect of human activities depends on quantifying the microbial communities in all ocean basins and understanding their community structure and function in response to natural and anthropogenic perturbations. Marine microbes are the foundation of the planet's trophic networks and play a critical role in planetary biogeochemical processes. They are the sentinels of the sea and respond rapidly to perturbations (e.g., Deepwater Horizon \cite{hazen_deep-sea_2010}).

\item{\em Basic physical parameters.} Temperature and conductivity, coupled with depth, reveal the hidden structure of ocean currents. Much of these structures cannot be directly observed with satellite and other remote-sensing technologies.

\item{\em Surface weather conditions.} Most of the world's oceans are not covered by the sophisticated Doppler radar systems used in terrestrial weather forecasting. Weather satellites can reveal a great deal by observing clouds from above but have limited ability to directly observe conditions underneath them. Simple rainfall observations and observations of sea surface and wave heights would be helpful for modeling.

\item{\em Debris sightings.} The abundance and trajectory of debris in the ocean have a large impact on marine ecosystems. When it is possible to track debris, it also can be used to monitor the evolution and status of ocean currents.

\end{itemize}